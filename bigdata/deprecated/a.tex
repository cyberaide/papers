\documentclass{article}

%%%%%%%%%%%%%%%%%%%%%%%%%%%%%%%%%%%%%%%%%%%%%%%%%%%%%%%%%%%%%%%%%%%%%%
% LATEX DEFINITIONS
%%%%%%%%%%%%%%%%%%%%%%%%%%%%%%%%%%%%%%%%%%%%%%%%%%%%%%%%%%%%%%%%%%%%%%

\usepackage{hyperref}
\usepackage{array}
\usepackage{graphicx}
\usepackage{booktabs}
\usepackage{pifont}
\usepackage{todonotes}
\usepackage{rotating}
\usepackage{color}
\usepackage{afterpage}
\usepackage{amssymb}
\usepackage{latexsym}

\newcommand*\rot{\rotatebox{90}}

\newcommand{\FILE}[1]{\todo[color=green!40]{#1}}

%%%%%%%%%%%%%%%%%%%%%%%%%%%%%%%%%%%%%%%%%%%%%%%%%%%%%%%%%%%%%%%%%%%%%%
% TITLE ABLE OF CONTENTS
%%%%%%%%%%%%%%%%%%%%%%%%%%%%%%%%%%%%%%%%%%%%%%%%%%%%%%%%%%%%%%%%%%%%%%

\newcommand{\TITLE}{The FutureGrid Testbed for Big Data}
\newcommand{\AUTHOR}{Gregor von Laszewsi, Geoffrey C. Fox}
\newcommand{\EMAIL}{laszewski@gmail.com}
\begin{document}

\title{\TITLE}
\author{\AUTHOR}
\date{\EMAIL}


%%%%%%%%%%%%%%%%%%%%%%%%%%%%%%%%%%%%%%%%%%%%%%%%%%%%%%%%%%%%%%%%%%%%%%
% TABLE OF CONTENTS
%%%%%%%%%%%%%%%%%%%%%%%%%%%%%%%%%%%%%%%%%%%%%%%%%%%%%%%%%%%%%%%%%%%%%%

\pagenumbering{roman}

\begin{center}
{\Large\bf \TITLE}\\
{\AUTHOR}\\
{\EMAIL}
\end{center}

\tableofcontents

\newpage

%%%%%%%%%%%%%%%%%%%%%%%%%%%%%%%%%%%%%%%%%%%%%%%%%%%%%%%%%%%%%%%%%%%%%%
% LIST OF TODOS
%%%%%%%%%%%%%%%%%%%%%%%%%%%%%%%%%%%%%%%%%%%%%%%%%%%%%%%%%%%%%%%%%%%%%%

\listoftodos

\newpage

%%%%%%%%%%%%%%%%%%%%%%%%%%%%%%%%%%%%%%%%%%%%%%%%%%%%%%%%%%%%%%%%%%%%%%
% TITLE OF PAPER
%%%%%%%%%%%%%%%%%%%%%%%%%%%%%%%%%%%%%%%%%%%%%%%%%%%%%%%%%%%%%%%%%%%%%%

\pagenumbering{arabic}

\maketitle

%%%%%%%%%%%%%%%%%%%%%%%%%%%%%%%%%%%%%%%%%%%%%%%%%%%%%%%%%%%%%%%%%%%%%%
% ABSTRACT OF PAPER
%%%%%%%%%%%%%%%%%%%%%%%%%%%%%%%%%%%%%%%%%%%%%%%%%%%%%%%%%%%%%%%%%%%%%%


\begin{abstract}

In this chapter we will be introducing you to FutureGrid that provides a testbed to conduct research for Cloud, Grid, and High Performance Computing. Although FutureGrid has only a modest number of compute cores (about 4500 regular cores and 14000 GPU cores) it provides an ideal playground to test out various frameworks that may be useful for users to consider as part of their big data analysis pipelines. 

The chapter is structured as follows. First we will provide the reader with an introduction to Future Grid. We will list a number of projects that use Futuregrid to conduct data analysis and introduce some of them to the reader. We will tell you about which services and Hardware exists. Next we will analyze which services are preinstalled and are available for big data analysis. As services that users may need for their work we point out how such a testbed can be utilized not only while provisioning virtual machines, but also on bare metal. 

We conclude the chapter with our observation cast throught three years of operating FutureGrid and provide an outlook for the next steps.
\end{abstract}

%%%%%%%%%%%%%%%%%%%%%%%%%%%%%%%%%%%%%%%%%%%%%%%%%%%%%%%%%%%%%%%%%%%%%%
% SECTIONS
%%%%%%%%%%%%%%%%%%%%%%%%%%%%%%%%%%%%%%%%%%%%%%%%%%%%%%%%%%%%%%%%%%%%%%

\FILE{introduction.tex}

\section{Introduction}

FutureGrid \cite{las2010gce,las12fg-bookchapter}``is a project led by Indiana University and funded by the National Science Foundation (NSF) to develop a highperformance grid test bed that will allow scientists to collaboratively develop and test innovative approaches to parallel, grid, and cloud computing. FutureGrid will provide the infrastructure to researchers that allows them to perform their own computational experiments using distributed systems. The goal is to make it easier for scientists to conduct such experiments in a transparent manner.  FutureGrid users will be able to deploy their own hardware and software configurations on a public/private cloud, and run their experiments. They will be able to save their configurations and execute their experiments using the provided tools. The FutureGrid test bed is composed of a highspeed network connecting distributed clusters of high performance computers. FutureGrid employs virtualization technology that will allow the test bed to support a wide range of operating systems.''



\section{Overview of FutureGrid}

\FILE{hardware.tex}

\subsection{Hardware Overview}

According to the manual, FutureGrid is build out of a number of clusters of different type and size that are interconected with up to a 10GB Ethernet among its sites. The sites include Indiana University, University of Chicago, San Diego Supercomputing Center, Texas Advanced Computing Center, and University of Florida.

\subsubsection{Overview of the Clusters}\label{S:hw-cluster} 

\begin{table}[htb]

\caption{FutureGrid Compute Resources}\label{T:hw}

\begin{center}
\begin{tabular}{rrrrrrrrr}
Name    & System Type                &  \rot{Nodes} &  \rot{CPUS}   & \rot{Cores}   & \rot{TFLOPS}  & \rot{RAM (GB)}        & \rot{Storage (TB)}    & Site \\
\hline
india   & IBM iDataplex              & 128          & 256     & 1024    & 11      & 3072            & 335             & IU \\
hotel   & IBM iDataplex              & 84           & 168     & 672     & 7       & 2016            & 120             & UC \\
sierra  & IBM iDataplex              & 84           & 168     & 672     & 7       & 2688            & 96              & SDSC \\
foxtrot & IBM iDataplex              & 32           & 64      & 256     & 3       & 768             & 0               & UF \\
alamo   & Dell Poweredge             & 96           & 192     & 768     & 8       & 1152            & 30              & TACC \\
xray    & Cray XT5m                  & 1            & 166     & 664     & 6       & 1328            & 5.4             & IU \\
bravo   & HP Proliant                & 16           & 32      & 128     & 1.7     & 3072            & 128             & IU \\
delta   & \shortstack{SuperMicro\\ GPU Cluster}     & 16           & 32      & 192     &         & 1333            & 144             & IU \\
lima    & Aeon Eclipse64             & 8            & 16      & 128     & 1.3     & 512             & 3.8             & SDSC \\
echo    & \shortstack{SuperMicro \\ScaleMP Cluster} & 16           & 32      & 192     & 2       & 6144            & 192             & IU \\
\end{tabular}
\end{center}
\end{table}


\FILE{hw-table.tex}

\begin{sidewaystable}

\caption{FutureGrid cluster details.}\label{F:cluster-details}
~\\
\begin{footnotesize}
\begin{tabular}{|p{2cm}||p{4cm}p{1.5cm}p{1.5cm}p{1.5cm}p{1.5cm}p{1.5cm}p{1.5cm}p{1.5cm}p{1cm}|}
\hline
 \bf Name                                & \bf Echo & \bf Alamo & \bf Bravo & \bf Delta & \bf Foxtrot & \bf Hotel & \bf India & \bf Sierra & \bf Xray \\
\hline
\hline
 Organization                        & IU & TACC & IU & IU & UF & UC & IU & SDSC & IU \\
\hline
 Machine Type                        & Cluster SclaeMP & Cluster & Cluster & Cluster & Cluster & Cluster & Cluster & Cluster & Cluster \\
\hline
 System Type                         &SuperMicro& Dell PowerEdge M610 Blade & HP Proliant && IBM iDataPlex dx 360 M2 & IBM iDataPlex dx 360 M2 & IBM iDataPlex dx 360 M2 & IBM iDataPlex dx 340 & Cray XT5m \\
\hline
 CPU Type                            & Xeon E5-2640 &  Xeon X5550 &  Xeon E5620 &  Xeon 5660 &  Xeon X5520 &  Xeon X5550 &  Xeon X5550 &  Xeon L5420 & Opteron 2378 \\
\hline
 CPU Speed                           &2.50GHz& 2.66GHz & 2.40GHz & 2.80 GHz & 2.26GHz & 2.66GHz & 2.66GHz & 2.5GHz & 2.4GHz \\
\hline
 CPUs                                &&192&32&32&64&168&256&168&168 \\
\hline
 Servers                             &12&96&16&16&32&84&128&84&1 \\
\hline
 RAM                                 && 12GB DDR3 1333Mhz & 192GB DDR3 1333Mhz & 192GB DDR3 1333 Mhz & 24GB DDR3 1333Mhz & 24GB DDR3 1333Mhz & 24GB DDR3 1333Mhz & 32GB DDR2-667 & 8GB DDR2-800 \\
\hline
 Total RAM                           &&1152GB&3072GB&3072GB&768GB&2016GB&3072GB&2688GB&1344GB \\
\hline
 Number of cores                     &144&768&128&&256&672&1024&672&672 \\
\hline
 Tflops                              &&8&1.7&&3&7&11&7&6 \\
\hline
 Disk Size (TB)                      &2.8&48&&15&20&120&335&72&335 \\
\hline
 Hard Drives                         && 500GB 7.2K RPM SAS & 6x2TB 7.2K RPM SATA & 92GB 7.2K RPM SAS2 & 500GB 7200 RPM SATA & 1 TB 7200 RPM SATA & 3000GB 7200 RPM SATA & 160GB 7200 RPM SATA Drive & 6TB Lustre \\
\hline
 Shared Storage                      && NFS & NFS &NFS& NFS & GPFS & NFS & ZFS 82.2TB & NFS \\
\hline
 Interconnect                        && Mellanox 4x QDR IB & Mellanox 4x DDR IB &&& Mellanox 4x DDR IB & Mellanox 4x DDR IB & Mellanox 4x DDR IB & Cray SeaStar \\
\hline
\end{tabular}
~\\
IB = InfiniBand, Xenon = INtel Xenon, Opteron = AMD Opteron 

\end{footnotesize}

\end{sidewaystable}



\subsubsection{Overview of Networking}

The significant number of distinct systems within FutureGrid provide a
heterogeneous distributed architecture and are connected by
high-bandwidth network links supporting distributed system research
\cite{las12fg-bookchapter}. FutureGrids network used to have a
dedicated network between sites \cite{las12fg-bookchapter}. However,
the network infrastructure has recently changed due to changes as part
of its major network operator the National Lambda Rail.  Due to these
changes the operation of the network between the sites conducted by
the national lambda rail has switched opertaion to XSEDE and are no
longer exclusive. This is no major handicap for the projects conducted
on FutureGrid based on our project portfolio.  The current high level
network diagram is depicted in Figure~\ref{F:network}.

The core resources to FutureGrid at SDSC, IU, TACC, and UF are now all
connected via the XSEDE network and integrated via the FG core router
in Chicago. WIthin The IU network additional clusters are integrated
and are described in more detail in Section~\ref{S:hw-cluster}. 

A Spirent H10 XGEM Network Impairment emulator
\cite{www-network-impairment} can be colocated with resources at
Indiana University, to enable experiments to include network latency,
jitter, loss, and errors to network traffic.

In addition we have added several shaded components that are related
to a spcial software service called cloudmesh that we explain in more
detail in Section~\ref{S:cloudmesh}.


\begin{figure}[htb]
  \centering
    \includegraphics[width=1.0\textwidth]{images/fg-network-2014-cm.pdf}
  \caption{High level network diagram and conceptual integration of cloudmesh resources.}
\label{F:network}
\end{figure}

\subsubsection{Overview of Storage}

FutureGrid has only a very limited amount of storage space and users
are requested to remove their storage space after use. GutureGrid does
not provide capacity for long term storage or long term
experiments. Users with special needs may be acomodated by special
storage setups. The list of storage services is shown in Table \ref{T:storage}.

\begin{table}[htb]
\caption{Storage Resources of FutureGrid.}
\label{T:storage} 

\centering{}%
\begin{tabular}{lrll}
\textbf{System Type } & \textbf{Capacity(TB) } & \textbf{File System } & \textbf{Site }\tabularnewline
\hline 
Xanadu 360  & 180  & NFS  & IU \tabularnewline
DDN 6620  & 120  & GPFS  & UC \tabularnewline
Sunfire x4170  & 96  & ZFS  & SDSC \tabularnewline
Dell MD3000  & 30  & NFS  & TACC \tabularnewline
IBM dx360 M3  & 24  & NFS  & UF \tabularnewline
\end{tabular}
\end{table}



\todo{services.tex}
\section{Services and Tools for Big Data}

FutureGrid offers a very rich environment to its users. We can categorize them in a stacked service architecture as depicted in 
Figure {F:services}. We distinguish the following categories: Cloud PaaS, IaaS, GridaaS, HPCaaS, TestbedaaS which we will explain in more detail in the next sections. These services are integrated in our general FG high-level architecture depicted in \ref{F:arch}. More detail about the architecture can be found in \cite{las2010gce,las12fg-bookchapter}. Within this paper we will focus on describing services that have been explicitly used for big data research in FutureGrid.

\begin{figure}[p]
  \centering
    \includegraphics[width=0.9\textwidth]{images/user-services.pdf}
  \caption{FutureGrid high-level user services.}\label{F:services}
  ~\\
  \centering
  \includegraphics[width=0.9\textwidth]{images/architecture.pdf}
  \caption{FutureGrid high-level architecture.}\label{F:arch}.
\end{figure}

\subsection{Testbed as a Service (TestbedaaS)}

It is a well accepted paradigm that today a lot of research including the one performed in big data are carried out by interdisciplinary scientific teams. Thus, FutureGrid provides a nice framework to manage user and project affiliation and propagates this information to a variety of subsystems constituting the FG service infrastructure. This includes operational services (not explicitly mention in Figure label{F:services}) to deal with authentication, authorization and accounting. In particular we have developed a unique metric framework that allows us to create usage reports from all of our Infrastructure as a Service frameworks. Repeatable experiments can be created with a number of tools including Pegasus, Precip and Cloudmesh. VMs can be managed on high level either via cloudmesh (see Section ?) or phantom. Provisioning of services and images can be conducted by RAIN \cite{?} and cloudmesh (see Section ?). Infrastructure monitoring is enabled via Nagios \cite{nagios}, Ganglia \cite{ganglia}, and Inca \cite{inca}.

\subsection{Traditional High Performance Computing as a Service (HPCaaS)}

Within the traditional high performance computing services FG offers a traditional MPI/batch queuing system and a virtual large memor system that are beneficial for big data calculations.

\subsubsection{MPI and Batch Queues}

The traditional high performance computing environment provided by queuing systems and Message Passing Interface (MPI) programs provide a suitable infrastructure not only for simulations, but also for the analysis of large data. However, considerable amount of work has to be put in place to optimize the available infrastructure for the problem domain. This has been successfully demonstrated for many biological applications. Additionally the existence of a queuing system can provide some advantages when the available resources are utilized to a full extend and resource starvation exists while sharing the resources with other users. This has been especially useful to also support educational activities for clases with many users that for example want to test map reduce activities controlled by a queuing system as described in Section \ref{S:hadoop}.

\subsubsection{Virtual Large-Memory System}

One of the demands often posed in big data analysis it to place the data as much as possible into memory to speed up calculations and in some cases to fit the entire dataset into memore. However, this analysis may come at a cost as for example the use of HPC computing via MPI adds additional programming complexity within a cluster. Therefore it is desirable to virtualize the memory from multiple servers in a cluster to provide one big memory system that can be easily accessed by the underlying software. 
One such implementation, vSMP by ScaleMP \cite{www-scalemp}.
Experiments conducted on futureGrid using HPCC
benchmarks show only a 4-6\% drop in efficiency when compared to native
cluster performance \cite{las12fg-bookchapter}. This makes it feasible for many applications. ScaleMP is installed on the FutureGrid echo cluster that has 16 servers and can access up to 3TB in shared virtual memory.

\subsection{Grid as a Service (GridaaS)}

Not surprisingly the demand for computational Grids on FutureGrid has been relatively small. While we saw few requests for Globus we decided to focus on the installation of more popular systems. The little use can be explained by the availability of large Grid production infrastructure elsewhere such as in XSEDE and based on the move of the community away from complex Grid solutions to either cloud computing or even back to more traditional batch processing solutions.


\subsection{Infrastructure as a Service (IaaS)}

One of the main features of FutureGrid is to offer its users a variety of infrastructure as a service frameworks. These frameworks provide virtualized resources to the users on top of existing cyberinfrastructure fabric. This includes but is not limited to virtualized servers, storage, network, disk, and load balancers. In FutureGrid the most common hypervisor that runs the virtual machines as guest on the underlaying operating system is KVM. Some resources also run XEN. Through the ability to provide large numbers of virtual machines to the users, access mode to utilize the resources has been changed from a reservation based service to an on-demand service. This comes with the benefit that if enough resources are available they will be immediately allocated to the user. However if not enough resources can be offered the system will decine the request and return with an error. Based on our experience with FG over the last couple of years it is advantageous to offered a mixed operation model. This includes a standrad production cloud that operates on-demand, but also a set of reserved cloud instances that can be reserved for a particular project. We have conducted this for several projects in FutureGrid including those that required dedicated access to resources as part of big data research such as classes \cite{fg405,fg368} or research projects with extremely large virtual machines \cite{fg298}.

The IaaS services that are offered in FutureGrid contain the following:

\begin{description}
\item [OpenStack] has become most recently besides HPC the most requested services in FG based on newly started projects. OpenStack is an open source cloud infrastructure as a service framework to deliver public and private clouds. It contains a number of components that together build a powerful and flexible set to create a cloud service offering. Services include a compute service, and object storage, an image service, a monitoring service, and an orchestration service. OpenStack has received considerable momentum due to its openness and the support of companies. Within FutureGrid OpenStack clouds are currently deployed on india, sierra, hotel, and alamo.  

\item [Nimbus] is an opensource service package allowing users to run virtual machines on FutureGrid hardware. Just as in Openstack users can upload their own virtual machine images or customize existing once. Nimbus, next to Eucalyptus is one of the earlier frameworks that make managing virtual machines possible. Nimbus provides a basic set of cloud services including services to easier orchestrate a cloud setup. However, such services are now also provided by Eucalyptus and OpenStack. Nimbus does not provide as of this writing features for project management. Nimbus provides a selected subset of AWS protocols such as EC2. Accounting is done on a user-by-user basis. This has some implications on user management as in large scale deployments project management features are highly desired.

\item [Eucalyptus] is an opensource software IaaS framework for cloud computing. Eucalyptus provides an Amazon Web Services (AWS) compliant EC2-based web service interface to its users enabling the easy integration between a local cloud managed with Eucalyptus and AWS. However, as others such as OpenStack also provide EC2 interfaces for many application users OpenStack has become a viable alternative.

\end{description}

Which of the IaaS frameworks to chose is a question that is not that easy to answer. Many of our projects evaluate multiple of them to chose the one best suited for their use case. At other times users chose a framework that they have previously successfully used. Over time the quality of the IaaS framework has significantly changed. Within the last year Openstack has become the most popular  platform on FutureGrid.


\subsection{Cloud Platform as a Service (PaaS)}


\section{Hadoop}



From http://hadoop.apache.org

''The Apache Hadoop project develops opensource software for reliable, scalable, distributed computing.

The Apache Hadoop software library is a framework that allows for the distributed processing of large data sets across clusters of computers using simple programming models. It is designed to scale up from single servers to thousands of machines, each offering local computation and storage. Rather than rely on hardware to deliver highavailability, the library itself is designed to detect and handle failures at the application layer, so delivering a highlyavailable service on top of a cluster of computers, each of which may be prone to failures.
'' \cite{www/hadoop}

from sriram \cite{report/myhadoop}

\section{Hadoop}

Traditional HPC environments typically support batch job submissions using resource
management systems such as the TORQUE Resource Manager (also known as the
Portable Batch System – PBS) or the Sun Grid Engine (SGE). On the other hand, Hadoop
provides it own scheduling, and manages its own job and task submissions, and tracking.
Since both systems are designed to have complete control over the resources that they
manage, the challenge is how to enable users to run Hadoop jobs in a typical HPC
environment using a scheduler such as PBS or SGE. In this release, we support Hadoop
job submissions via PBS and SGE. However, this approach is equally feasible for other
schedulers such as Condor, as well.
Our approach is to configure Hadoop clusters “on-demand” by first requesting resources
for an Nnode Hadoop cluster via PBS. Once the resources are received, the Hadoop
configurations and environments are set up based on the set of resources provided by
PBS. The Hadoop Distributed File System (HDFS) can be configured in one of two ways
– in 1) transient (nonpersistent) or 2) persistent modes. In the nonpersistent mode, the
HDFS is set up to use local storage. In the persistent mode, the HDFS is set to
symbolically link to an external location that will be persistent – i.e. data from Hadoop
runs will continue to persist even after the Hadoop runs are complete. More details are as
follows.

\subsection{Details}

The prerequisite for myHadoop is a valid Hadoop installation – we recommend that you
use Hadoop version 0.20.2 since that is the only version of Hadoop that this package has
been tested with. Henceforth, we will refer to the location of the Hadoop installation as
HADOOP\_HOME. We will refer to the location of the myHadoop installation (i.e. this
package) as MY\_HADOOP\_HOME. The \$MY\_HADOOP\_HOME/pbs-example.sh shows
an example of how to use myHadoop with PBS. A similar script for SGE can be found in
\$MY\_HADOOP\_HOME/sge-example.sh.
A step-by-step process for using myHadoop is as follows.

\subsection{Initial Configuration}

Ensure that the environment variables inside \$MY\_HADOOP\_HOME/bin/setenv.sh are
set correctly. You can set your HADOOP\_HOME, and the locations for your HDFS data
and log directories using this script. You will need to update this script before you can
proceed further.
All the tuning parameters for Hadoop can be found in the \$MY\_HADOOP\_HOME/etc
directory. There is no need to edit any of the parameters, especially if you are not an
expert Hadoop user. If you are familiar with the various Hadoop parameters, you may
edit the parameters that fall outside the “DO NOT EDIT” sections.

\subsection{Request N nodes from the Scheduler}

Once the environment variables have been set correctly, we are ready to use myHadoop
using a regular PBS or SGE submission script. Your PBS script should contain the
following lines to initialize PBS as follows:

\begin{verbatim}
#!/bin/bash
#PBS -q <queue_name>
#PBS -N <job_name>
#PBS -l nodes=4:ppn=1
#PBS -o <output file>
#PBS -e <error_file>
#PBS -A <allocation>
#PBS -V
#PBS -M <user email>
#PBS -m abe
\end{verbatim}

In the above case, we are requesting 4 nodes. Note that you must set the processors per
node (ppn) to 1.
Your SGE script should contain the following lines to initialize SGE:

\begin{verbatim}
#!/bin/bash
#$ -V -cwd
#$ -N <job_name>
#$ -pe <queue_name> 4
#$ -o <output file>
#$ -e <error file>
#$ -S /bin/bash
\end{verbatim}

For SGE, there is one important rule to remember. The queue name specified above
should be preconfigured with an allocation\_rule set to 1 (one). This ensures that the
Hadoop cluster is set up such that multiple instances of the Hadoop daemons are not
scheduled on the same node.

\subsection{Set the myHadoop Environment}

Run the \$MY\_HADOOP\_HOME/bin/setenv.sh script (that you modified in Section 2.1)
to set all the environment variables required by myHadoop.
. \$MY\_HADOOP\_HOME/bin/setenv.sh
Set the HADOOP\_CONF\_DIR to the directory where Hadoop configs should be
generated – all configuration files for the Hadoop run will be picked up from here. Ensure
that this directory is accessible to all nodes – and a way to do this is to make sure that this
directory is on a shared file system such as NFS or Lustre.
export HADOOP\_CONF\_DIR=<configuration directory>

\subsection{Configure the myHadoop Cluster}

You can initialize and configure the Hadoop cluster by using the
\$MY\_HADOOP\_HOME/bin/pbs-configure.sh (or sge-configure.sh) script. You may
create a transient or persistent myHadoop cluster by changing the commandline
arguments as follows.
For a transient myHadoop cluster, configure it as follows (replace 4 with the total number
of nodes requested):
\$MY\_HADOOP\_HOME/bin/pbs-configure.sh -n 4 -c \$HADOOP\_CONF\_DIR
In this mode, you will have to copy all of your data into the myHadoop cluster after it is
configured, and copy out the results after the job is complete. All data will be
inaccessible from HDFS once the PBS job is complete.
Alternatively, you may set up a persistent myHadoop cluster by using the –p option, and
setting the BASE\_DIR for HDFS as follows:
\$MY\_HADOOP\_HOME/bin/pbs-configure.sh -n 4 -c \$HADOOP\_CONF\_DIR -p -d
<HDFS BASE\_DIR>
The BASE\_DIR should be on a directory accessible to all nodes, to ensure that the data
will not be cleaned up after job completion. For instance, the BASE\_DIR could be on a
Lustre file system. Note that, if N-node cluster is being created, then the BASE\_DIR
should have directories named 1, 2, … , N. The configuration script sets up symbolic
links from node I to the BASE\_DIR/I directory. When this mode is used, there is no need
to copy data back and forth from HDFS to another file system between runs.

\subsection{Format HDFS (if need be)}

If myHadoop is being used in transient mode, or if it is being used for the first time in
persistent mode, then you will have to format the HDFS as follows:
\$HADOOP\_HOME/bin/hadoop --config \$HADOOP\_CONF\_DIR namenode –format

\subsection{Run Hadoop Jobs}

You are now all set to start all the Hadoop daemons as follows:
\$HADOOP\_HOME/bin/start-all.sh
Once the daemons are all started up, you can start using Hadoop as usual. You may also
stage data in and out from HDFS, as required.

\subsection{Clean up}

Although, PBS or SGE may be set up to automatically clean up after your Hadoop job is
complete, it is always a good idea to stop all the Hadoop daemons, and use the cleanup
script to clean up after yourself.
\$HADOOP\_HOME/bin/stop-all.sh
\$MY\_HADOOP\_HOME/bin/pbs-cleanup.sh -n 4 OR
\$MY\_HADOOP\_HOME/bin/sge-cleanup.sh -n 4


\subsection{Hadoop}

My Hadoop

We have various platforms that support Hadoop on FutureGrid. MyHadoop is probably the easiest solution offered for you. It provides the advantage that it is integrated into the queuing system and allows hadoop jobs to be run as batch job. This is of especial interest for classes that may run quickly out of resources if every students wants to run their hadoop application at the same time.



MapReduce is a programming model developed by Google. Their definition of MapReduce is as follows: “MapReduce is a programming model and an associated implementation for processing and generating large data sets. Users specify a map function that processes a key/value pair to generate a set of intermediate key/value pairs, and a reduce function that merges all intermediate values associated with the same intermediate key.” For more information about MapReduce, please see the Google paper here.

The Apache Hadoop Project provides an open source implementation of MapReduce and HDFS (Hadoop Distributed File System).

This tutorial illustrates how to run Apache Hadoop thru the batch systems on FutureGrid using the MyHadoop tool.

\subsubsection{myHadoop on FutureGrid}

MyHadoop is a set of scripts that configure and instantiate Hadoop as a batch job.

myHadoop 0.20.2 is currently installed on Alamo, Hotel, India, and Sierra FutureGrid systems.



\FILE{usage.tex}

\section{FutureGrid Usage}

When offering services such as FutureGrid to the community, we have to
analyse and predict which services may be useful for the users. We
have therefore established a number of activities that monitor
external and internal data. Externally, we look for example at
information provided by Gartners technology hype curve \cite{?} or
Google trend information as shown in Figure \ref{F:google-trend}. From
Google Trend data we observe that the popularity of Grid computing
has been very low  in the recent years and much attention has shifted
to cloud computing. Therefore we removed this information from the
figure and focus exemplary on cloud related terms such as {\em Cloud
  Computing}, {\em Big Data}, {\em OpenStack} {\em VMWare}.
From this information we see that all but VMWare are rising, with
Cloud Computing dominating the google trends in comparison to the
others. This trend is important as it shows a shift in the cloud
computing community buzz away from a traditional commercial market
leader in virtualization technology. We believe that is correlated
with a large number of vendors offering alternative products and
services while at the same time the novelty from VMWare is reduced.

\begin{figure}[htb]
 \centering
    \includegraphics[width=.75\textwidth]{images/google-trend.pdf}
  \caption{Google Trends.}\label{F:google-trend}
\end{figure}

To give an informal overview of the more than 300 projects conducted
on FutureGrid we have taken their titles and display them in a word
cloud (see Figure \ref{F:wordcloud}. Additionally, we have taken
keywords that are provided by the project leads and also displayed
the in a word cloud (see Figure \ref{F:keycloud}. Although the
images doe not give quantitative perspective about the project it helps
to identify some rough idea about the activities that are ongoing in FutureGrid.
As expected the terms cloud computing an and terms such as mapreduce,
Openstack, Nimbus, and Eucalyptus appear quite frequently. It is hence
worthwhile to analyse this data in a more quantitative form.

\begin{figure}[p]
\begin{minipage}[t]{1.0\textwidth}
  \centering
    \includegraphics[width=1.0\textwidth]{images/fg-title-wordcloud.pdf}
  \caption{Project title word cloud.}\label{F:wordcloud}
\end{minipage}
\vspace{24pt}\\
\begin{minipage}[t]{1.0\textwidth}
  \centering
    \includegraphics[width=1.0\textwidth]{images/fg-keyword-wordcloud.pdf}
  \caption{Project keyword word cloud.}\label{F:keycloud}
\end{minipage}
\end{figure}


As part of our project management in FutureGird we have designed a
simple project application procedure that includes prior to a project
granted access gathering information about which technologies are
anticipated to be used within the project. The list is fairly
extensive and includes Grid, HPC, and Cloud computing systems,
services, and software. However, for this paper we will focus
primarily on technologies that are dominantly requested and depicted
in Figure~\ref{F:request-tech}. Clearly we can identify the trend,
that shows the increased popularity of OpenStack within the services
offered on FutureGrid. Nimbus and Eucalyptus are on a significant
downwards trend. ObenNebula was also at one point more requested that
either Nimbus or Eucalyptus, but do to limited manpower an official
version of OpenNebula was not made available. As we have not offered
it and pointed it out on our Web page, requests for OpenNebula have vanished.
However internally have used OpenNebula for projects such as our cloudmesh rain
framework. All other sixteen technologies are relatively equally
distributed over the monitoring period. The lesson that we took form
this is that FutureGrid has put more emphasize in offering OpenStack Services.

\begin{figure}[htb]
  \centering
    \includegraphics[width=1.0\textwidth]{images/trend-a.pdf}
  \caption{Requested technologies by project}\label{F:request-tech}
\end{figure}

\begin{figure}[p]
 \centering
    \includegraphics[width=.7\textwidth]{images/project-frequency.pdf}
  \caption{Project Frequency.}\label{F:project-members}

  \centering
    \includegraphics[width=.6\textwidth]{images/project-disciplines.pdf}
  \caption{Distributon of project disciplines.}
  \label{F:freq-dis}

  \centering
    \includegraphics[width=.75\textwidth]{images/trend-b.pdf}
  \caption{Requests by of technologies by discipline within a
    project. $\bigtriangleup$ = Map Reduce, Hadoop, or Twister,
    $\Box$  = MPI, $\circ$ = ScaleMP}
   \label{F:trend-b}
\end{figure}

From the overall project information we have also analysed the
frequency of the number of project members within the project and show
it in Figure~\ref{F:project-members}. Here we depict on the abscissa
classes of projects with varying members.  Assume we look at the
abscissa value of 10, This means that these are all projects that
have project members between 10 and its previous category in this case
5. Hence, it will be all projects greater  5 and smaller or equal
10. With this classification  we see that the dominant unique number of
members within all projects is either one, two or there members. Than
we have another class between 4 and 10 members, and the rest with more
than ten members. One of the projects had overall 186 registered
members, for an education class as part of a summer school. Looking at
the distribution of the members and associating them with research and
education projects, we find all projects with larger numbers of
projects to be education projects.



%Figure~\ref{F:bigdata-freq} shows the frequency distribution of technologies such as
%map/reduce, hadoop, and twister by year. In the chart we simply called
%the agglomoration of these technologies big data.  As we see relative
%to all projects requested within one year we identified a rising trend.

%\begin{figure}[htb]
%  \centering
%    \includegraphics[width=0.2\textwidth]{images/bigdata-freq.pdf}
%  \caption{Big Data Project frequency.}\label{F:bigdata-freq}
%\end{figure}

Next we have analyzed the for all projects that requested either mapreduce,
hadoop, twister, MPI and ScaleMP (147 of all 374 active projects,
which is 39\% of all projects) and categorized them by discipline as
shown in in \ref{F:freq-dis}. In contrast to XSEDE which provides a production HPC
system to the scientific community, the usage of FutureGrid is
dominated with 50\% by computer science related projects\footnote{is
  this for all projects or just within hadoop? This is not cear from
  the data I received.} Education is the next highest with 19\%.


If we look at further into this data we present in Figure
\ref{F:trend-b} The number of projects in a particular category, as
well as the Fraction of technologies within a discipline. As we are in
this paper interested in the impact on big data. we have looked in
particular at requests for mapreduce, hadoop, and twister, awhile also looking at
requests for MPI and ScaleMP. It is interesting to note that the
percentual distribution of the technologies among these projects is
about constant if we exclude technology evaluations and interoperability. As
MPI is more popular with domain sciences we find a slight increase in
projects requesting MPI. However with the life sciences we see the
oposite as mapreduce and associated technologies are more popular
here. MPI and ScaleMP are not much requested as part of technology
evaluations and interoperability experimentation is as they either
project a very stable framework and does not require evaluation, or
the question of interoperability is not of concern for most of the projects.
\footnote{slash box does not show up, do we need another latex package}




%\begin{figure}[htb]
%  \centering
%    \includegraphics[width=1.0\textwidth]{images/project-member-dist.pdf}
%  \caption{Project Member Dist.}
%\end{figure}

%\afterpage{\clearpage}

\FILE{devops.tex}

\section{System Management}

The goals of FutureGrid to offer a variety of services as part of its
testbed features is going beyond services that are normally offered by
data and supercomputing centers for research. This provides a number
of challenges that need to be overcome in order to efficiently manage
the system and provide services that have never been offered to users
as they exist on FutureGrid.

\subsection{Integration of Systems and Development Team}

FutureGrid started initially with a model where the systems team and
the software team were separated. A clear wall of responsibilities was
erected that resulted in multiple challenges:

\begin{enumerate}

\item
The system setup and management was completely separated from
the software development team focussing mostly on the deployment of
existing technologies. 

\item
The system was complex, but its deployment was documented to a limited
extend not allowing the developers to utilize it properly.

\item 
Lack of trust by the systems team dis not allow the software team to
have a valid development environment.

\item
The software developed needed a testbed within the
testbed that was not necessarily reflecting the actual system setup.

\end{enumerate}

Together these issues, made it extremely difficult if not impossible to
further any development in regards to the design of a testbed
infrastructure as requested by our original ambitious goals. 

To overcome these difficulties it was decided that the systems team
must be integrated in some fashion into the software team and become
part of the development process. This integration is not an isolated
instance within FutureGrid, but is also executed in many modern data
centers and is now recognized with its own term called {\em DevOps}.

\subsection{DevOps}

DevOps is not just a buzzword from industry and research communities,
but it provides value added processes to the deployment and
development cycles that are part of modern data centers. It can today
be understood as a software development method that stresses
collaboration and integration between software developers and
information technology professionals such as a system administrator. 

While using an infrastructure such as clouds we recognized early on
that the lifetime of a particular IaaS framework is about 3-6 month
before a new version is installed. This is a significant difference to
a traditional High Performance Computing Center that is comprised of
many software tools experiencing much longer life spans. This is not
only based on security patches but significant changes for example in
the evolving security, user services as well as the deployment of new
services that become available in rapid procession.

This rapid change of the complex infrastructure requires a rethinking
about how systems in general are managed and how they can be made
available to the development teams. While previously it may have been
enough to install updates on the machines, DevOps frameworks provide
the developer and system administrators to create and share
environments that are used in production and development while at the
same time increasing quality assurance by leveraging each others
experiences (see Figure \ref{F:devops}).

\begin{figure}[htb]
  \centering
   \begin{minipage}{.5\textwidth}
    \includegraphics[width=1.0\textwidth]{images/devops.pdf}
    \caption{DevOps Intersection.}
    \label{F:devops}
  \end{minipage}%
   \begin{minipage}{.5\textwidth}
     \includegraphics[width=1.0\textwidth]{images/devops-circle.pdf}
     \caption{DevOps Cycle.}
     \label{F:devops-circel}
  \end{minipage}%
\end{figure}

\subsubsection{DevOps Cycle}

While combining the steps executed by the Development and operational
team from planing, to coding, building and testing, to the release,
deployment and operation and monitoring (see Figure
\ref{F:devops-circle}), each of the phases provides a direct feedback
between the DevOps team members and shortening thus the entire
development phase. It also allows to test out new services and
technologies in a rapid progression. Hence it is possible to roll out
new developments much faster into production. This leads to a much mor
rapid integrated cycle than without the correlation between
development and operation would be possible.

\subsubsection{DevOps Supporting Tools}

A number of tools are available that make the introduction of DevOps
strategies more efficient. The first is the need for an efficient
communication pathway to manage tasks not only between developers but
also between users. Thus the ideal system would provide a complete
integration of a project management system that allows to manage tasks
for both developers and operators, but also to easily integrate
tickets and transform them into tasks. In XSEDE and other
supercomputing centers a system called RT is typically used for user
ticket management. Other systems such as jira, mantis, and ??? are
often used to manage the software and systems related
tasks. Unfortunately, personal or organizational constraints prevent
often the integration of the two systems and additional overhead is
needed to move user tickets into tasks and the development
cycle. Within FutureGrid we experimented as part of our opensource
development extensively with jira as systems and ticketing system
reveling that newest development in such areas motivated by DevOps
teams lead to tools that support the overall cycle including users
(see Figure \ref{F:usedevops}). However, the integration of
FutureGrid within the overall much larger XSEDE effort did make it not
possible to switch from RT to jira for user ticket
management. To stress this user integration we term this framework
{\em UseDevOps}. Tools to integrate Development and Operation
deployment include puppet, chef, ansible, cfengine and bcfg2. While
FutureGrid started out with bcfg2 we have since than switched to other
tools due to their prevalence within the community. Chef, puppet, and
ansible have significant amount of traction. Due to expertise within
our group we currently explore chef and ansible. 

\begin{figure}[htb]
  \centering
    \includegraphics[width=0.5\textwidth]{images/usedevops.pdf}
  \caption{User Support integrated into DevOps leads to UseDevops.}
  \label{F:usedevops}
\end{figure}








%http://en.wikipedia.org/wiki/DevOps








\subsection{Support for Education}

To support the many educational and research projects on FutureGrid we have provided through a portal significant amount of material on how to use the discussed services. In addition, we realized that not every educational project has users with sophisticated computer usage and we hance provide for such projects the ability to a streamlined user interface arther than having the users fight with complex command line syntax and parameters. Thus we provided for a recent MOOC on big data taught with resources on FutureGrid the basic funtionality not only to start VMs as part of the IaaS framework, but also to deploy sophisticated images that contain preinstalled software and allow sorvices to be hosted by the users such as iPypthon, R and much more. This was implemented on top of OpenSTack while utilizing the newest OpenStack services such as heat. The management of the VMs and starting of the iPython server was controlled by a python application that provides the user with a menu system. Thus the management of them becam literraly, as easy as pressing 1, 2, 3, ... in the menu.
For other classes we also have provided completely separate OpenStack deployments as the teachers were afraid that students woudl not have enough resources. However we learned from this that the teachers overestimated the actual utilization of the project and many resources were not used. Based on this analysis we do now have a model to justify the creation more relaxed access policies and potentially justify that even classes shoudl be provided in the main region that FG provides. If resource contention would become an issue we coudl set aside a special region for a limited number of time.
Reconfiguration needs have alo arisen where one day a class may want o explore traditional MPI, while the next they want to experiment with Hadoop.
Furthermore, we identified that several users wanted to combine various cloud IaaS platforms in order to avoid resource overprovisioning, or were interested in combining all resources. 



\FILE{cloudmesh.tex}

\section{Cloudmesh\footnote{This section is als available in part from
  the cloudmesh Web pages and includes large protions of copied
  text. The text is publically available.}}\label{S:cloudmesh}

From the experience with FutureGrid we identified the need for a more
tightly integrated software infrastructure addressing the need to
deliver a software-defined system – encompassing virtualized and
bare-metal infrastructure, networks, application, systems and platform
software – with a unifying goal of providing Cloud Testbeds as a
Service (CTaaS). This system is termed cloudmesh to symbolize (a) the
creation of a tightly integrated mesh of services targeting multiple
IaaS farmeworks (b) the ability to federate a number of resources from
from academia and industry. This includes existing FutureGrid
infrastructure, Amazon Web Services, Azure, HP Cloud, Karlsruhe using
not only one IaaS framework but various. (c) The creation of an
environment in which it becomes more easy to experiment with platforms
and softwere services while assiting to deploy them more easily.
In addition to virtual resources, FutureGrid exposes bare-metal
provisioning to users, but also a subset of HPC monitoring
infrastructure tools. Services will be available through command line,
API, and Web interfaces.

\subsection{Functionality}

Cloudmesh provides due to its integrated services the ability to be an
onramp for other clouds. It also provides information services to
various system level sensors to give access to sensor and utilization
data. They internally can be used to optimize the system
usage. The provisioning experience from FutureGrid has taught us that
we need to provide the creation of new clouds, the repartitioning of
resources between services (cloud shifting), and the integration of
external cloud resources in case of over provisioning (cloud
bursting). As we deal with many IaaS we need an abstraction layer on
top of the IaaS framework. Experiment management is conducted with
workflows controoled in shells, Python/iPython, as well as systems
such as Heat, Accounting is supported through additional services such
as user management and charge rate management. Not all features are
yet implemented. Figure \label{F:cm-func} shows the main functionality
that we target at this time to implement.

\begin{figure}[h!]
  \centering
    \includegraphics[width=1.0\textwidth]{images/cm-functionality.pdf}
  \caption{CM Functionality.}\label{F:cm-func}
\end{figure}


\subsection{Architecture}

The three layers of the Cloudmesh architecture include a Cloudmesh
Management Framework for monitoring and operations, user and project
management, experiment planning and deployment of services needed by
an experiment, provisioning and execution environments to be deployed
on resources to (or interfaced with) enable experiment management, and
resources.

\begin{figure}[h!]
  \centering
    \includegraphics[width=1.0\textwidth]{images/cm-arch.pdf}
  \caption{CM Architecture.}
\end{figure}

\paragraph{System Monitoring and Operations.}

The management framework contains services to facilitate FutureGrid day-to-day operation, including federated or selective monitoring of the infrastructure. Cloudmesh leverages FutureGrid for the operational services and allows administrators to view ongoing system status and experiments, as well as interact with users through ticket systems and messaging queues to inform subscribed users on the status of the system.
The cloudmesh management framework offers services that simplify integration of resources in the FutureGrid nucleus or through federation. This includes, for user management, access to predefined setup templates for services in enabling resource and service provisioning as well as experiment execution. To integrate IaaS frameworks cloudmesh offers two distinct services:

(a) a federated IaaS frameworks hosted on FutureGrid,
(b) the availability of a service that is hosted on FutureGrid allowing the “integration” of IaaS frameworks through user credentials either registered by the users or automatically obtained from our distributed user directory.

For (b) several toolkits exist to create user-based federations, including our own abstraction level which supports interoperability via libcloud, but more importantly it supports directly the native OpenStack protocol and overcomes limitations of the EC2 protocol and the libcloud compatibility layer. Plugins that we currently develop will enable access to clouds via firewall penetration, abstraction layers for clouds with few public IP addresses and integration with new services such as OpenStack Heat. We successfully federated resources from Azure, AWS, the HP cloud, Karlsruhe Institute of Technology Cloud, and four FutureGrid clouds using various versions of OpenStack and Eucalyptus. The same will be done for OpenCirrus resources at GT and CMU through firewalls or proxy servers.
Additional management flexibility will be introduced through automatic cloud-bursting and shifting services. While cloud bursting will locate empty resources in other clouds, cloud shifting will identify unused services and resources, shut them down and provision them with services that are requested by the users. We have demonstrated this concept in 2012, moving resources for ~100 users to services that were needed based on class schedules. A reservation system will be used to allow for reserved creation of such environments, along with improvements of automation of cloud-shifting.

\paragraph{User and Project Services}

FutureGrid user and project services simplify the application processes needed to obtain user accounts and projects. We have demonstrated in FutureGrid the ability to create accounts in a very short time, including vetting projects and users – allowing fast turn-around times for the majority of FutureGrid projects with an initial startup allocation. Cloudmesh re-uses this infrastructure and also allows users to manage proxy accounts to federate to other IaaS services to provide an easy interface to integrate them.

\paragraph{Accounting and App Store}

To lower the barrier of entry Cloudmesh will be providing a shopping cart which will allow checking out of predefined repeatable experiment templates. A cost is associated with an experiment making it possible to engage in careful planning and to save time by reusing previous experiments. Additionally, the Cloudmesh App Store may function as a clearing-house of images, image templates, services offered and provisioning templates. Users may package complex deployment descriptions in an easy parameter/form-based interface and other users may be able to replicate the specified setup with.
Due to our advanced Cloudmesh Metrics framework we are in the position to further develop an integrated accounting framework allowing a usage cost model for users and management to identify the real impact of an experiment on resources. This will be useful to avoid overprovisioning and inefficient resource usage. The cost model will be based not only on number of core hours used, but also the capabilities of the resource, the time, and special support it takes to set up the experiment. We will expand upon the metrics framework of FutureGrid that allows measuring of VM and HPC usage and associate this with cost models. Benchmarks will be used to normalize the charge models.

\paragraph{Networking.}

We have a broad vision of resource integration in FutureGrid with systems offering different levels of control from “bare metal” to use of a portion of a resource. Likewise, we must utilize networks offering various levels of control, from standard IP connectivity to completely configurable SDNs as novel cloud architectures will almost certainly leverage NaaS and SDN alongside system software and middleware. FutureGrid resources will make use of SDN using OpenFlow whenever possible and the same level of networking control will not be available in every location.

\paragraph{Monitoring.}

To serve the purposes of CISE researchers, Cloudmesh must be able to access empirical data about the properties and performance of the underlying infrastructure beyond what is available from commercial cloud environments. To accommodate this requirement we have developed a uniform access interface to virtual machine monitoring information available for OpenStack, Eucalyptus, and Nimbus. In the future, we will be enhancing the access to historical user information. Right now they are exposed through predefined reports that we create on a regular basis. To achieve this we will also leverage the ongoing work while using the AMPQ protocol. Furthermore, Cloudmesh will provide access to common monitoring infrastructure as provided by Ganglia, Nagios, Inca, perfSonar, PAPI and others.

\begin{figure}[h!]
  \centering
    \includegraphics[width=.7\textwidth]{images/rainbow.pdf}
  \caption{Monitoring the Service distribution of FutureGrid with cloudmesh.}
\end{figure}

\subsection{Cloud Shifting}

We have already demonstrated via the RAIN tool in cloudmesh that it is
possible to easily shift resources between services. We are currently
expanding upon this idea and developing more easy to use user
interfaces that assit administrators and users through role and
project based authentication to move resources from one service to
another (see Figure \ref{F:shift}).

\begin{figure}[h!]
  \centering
    \includegraphics[width=1.0\textwidth]{images/shift2.pdf}
  \caption{Shifting resources makes it possible to offer flexibility
    in the service distribution in case of over or underprovisioning.}\label{F:shift}
\end{figure}

\subsection{Graphical User Interface}

Despite the fact that cloudmesh was originally a quite sophisticated
command shell and commandline tool, we have spend recently more time
in exposing this functionality through a conveneient Web
interface. Some mor popular views if this interface are depicted in
Figure \ref{F:instances} hinting on how easy it is with a single
button to create multiple VMs accross a variety of IaaS. Also nice is
that this not only includes resources at IU but also at external
locations. Pushing this easy management in a more sophisticated
experience for the user while associating one-click deployments that
include the ability to deploy virtual clusters, hadoop environments,
and other more elaborate setups wie provide an early prototype
screenshot in Figure \ref{F:oneclick}.

\begin{figure}[htb]
  \centering
    \includegraphics[width=.9\textwidth]{images/instances.pdf}
  \caption{Rainbow.}\label{F:instances}
  \centering
    \includegraphics[width=.9\textwidth]{images/oneclick.pdf}
  \caption{One click deployment of platforms and sophisticated
    services that could even spawn multiple resources.}\label{F:oneclick}
\end{figure}


\afterpage{\clearpage}

\section{Summary}

In this chapter we have described FutureGrid and focused on services
that are beneficial for big data analysis. Based on the discussion it
is clear that such a system is extremly complex but provides many
benefits of offering multiple services within the same
infrastructure. Performance experiments can thus be not only conducted
while conducting big data analysis in virtual amchines, but on a
variety of IaaS and PaaS enviroenments. Moreoverthes experiemnets can
directly be compared to bare metal provisioned services. Hence, users
can evaluate what impact such technologies have on their
codes. Comparisions of different progarmming farmeworks can be
achieved and future activities in regards to efficiency and usability
can be deducted. The lessons learned from FutureGrid are motivating a
toolkit cloudmesh that already today allows to manage virtual machines
on a avriety of infrastructure as a service frameworks. The easy
deployment of sophisticated setups with a one click deployment has
been validated as part of an infrastructure designed for a
MOOC. Furthermore the novel concept of shifting resources
\cite{las08federated-cloud} between
services to topport services that need more resources is a significant
contribution by cloudmesh. Image management and creation under
security restrictions \cite{fg-1295}  is
furthermore an important aspect. We will continue to develop the
cloudmesh environment and make it available to our users. 


\section*{Acknowledgement}

Some of the text published in this chapter is available form the
FutureGrid portal. The FutureGrid project is funded by the National
Science Foundation (NSF) and is led by Indiana University with
University of Chicago, University of Florida, San Diego Supercomputing
Center, Texas Advanced Computing Center, University of Virginia,
University of Tennessee, University of Southern California, Dresden,
Purdue University, and Grid 5000 as partner sites. This material is
based upon work supported in part by the National Science Foundation
under Grant No. 0910812. If you use FutureGrid and produce a paper or
presentation, we ask you to include the reference
\cite{las2010gce,las12fg-bookchapter}.

\bibliographystyle{IEEEtranS}
\bibliography{%
bib/references,%
bib/vonLaszewski-jabref,%
bib/image-refs,%
bib/cyberaide-cloud,%
bib/python}

%%%%%%%%%%%%%%%%%%%%%%%%%%%%%%%%%%%%%%%%%%%%%%%%%%%%%%%%%%%%
%\clearpage

%\appendix

%\FILE{overview.tex}

\section{Overview}

FutureGrid is a national-scale Grid, Cloud and HPC computing test-bed service of modest size that includes a number of computational resources at five distributed locations. FutureGrid experience and architecture is built around software defined systems at all levels of the stack shown in figure – encompassing VM and bare-metal infrastructure, networks and application, systems and platform software – with a unifying goal of providing Computing Testbeds as a Service. FutureGrid systems total 4704 cores divided into distributed general purpose clusters at Chicago, Florida, IU and TACC; a Cray XT5m at IU and four small specialized clusters supporting SSD (at SDSC), Large Disk Large memory (at IU) and NVIDIA GPU’s (IU). FutureGrid’s system model has grown in sophistication and now supports software-defined systems – encompassing virtualized and bare-metal infrastructure, networks, application, systems and platform software – with a unifying goal of providing Cloud Testbeds as a Service (CTaaS). Cloudmesh aggregates resources not only from FutureGrid, but also from OpenCirrus, Amazon, Microsoft Azure, and HP Cloud and GENI resources. Cloudmesh was originally developed in order to simplify the execution of multiple concurrent experiments on a federated cloud infrastructure and in addition to virtual resources, FutureGrid exposes bare-metal provisioning to users.

Users can apply for projects either at FutureGrid or XSEDE portals. A single account provides a user access to all FutureGrid machines and we can describe the type of work that can be done by looking at FutureGrid’s experience based on 3 and a half years of operation with 355 projects and 2178 users from 55 countries with 76\% from the USA. 51.8\% of these projects were mainly computer science (including middleware and cyberinfrastructures) research, 9.6\% technology evaluation, 13.2\% education, 9.6\% in life sciences and 11.5\% in other application domains. We looked in detail at the last 200 projects starting October 25 2011 to understand in more detail usage paradigms. 98 of these projects needed only virtual machine (VM) access and 54 requested both virtualized and non-virtualized nodes. Of the 48 projects not requesting VM’s, 8 were studying cloud technology like Hadoop and so 160 projects (80\%) were cloud related. 16 projects involved GPU access and 30\% of all projects used MapReduce in some way. The use of FutureGrid for education has been increasing and 21\% of projects in last 2 years have been for education dividing into 29 semester length classes and the 13 remaining split between REU training, Summer Schools, Tutorials, and workshops. Of 42 education requests, 36 were computer science, 3 application oriented and 3 mixed. These education classes covered areas like cloud computing, distributed systems, parallel computing, big data, data-intensive computing and datamining, business analytics, autonomic computing, cyberinfrastructure, storage, software carpentry, data centers and large scale infrastructure, MapReduce, high performance computing, networking, science clouds, and particular tools supported on FutureGrid.
Coming to the 136 research projects, 109 had a major CS component and 44 an application component with 17 of these jointly classified. Application projects included 18 from bioinformatics including genomics, radiology, cardiovascular simulation, surgery control, health sensors, iPlant cyberinfrastructure and text mining. Only 10 application projects had a simulation (major focus of most HPC systems) focus including combustion, CFD, subsurface modeling, climate, weather, ocean, environment, earthquakes and supply chains. Physical science and engineering data intensive applications (8 projects) include astronomy, particle physics, aerospace reliability, ocean observation, hydroinformatics, GIS and accelerator control. 6 social science projects include conflict resolution, disaster management using Twitter, optimization, political science and economics.

Turning to CS related projects, 23 were in basic virtualization (IaaS) areas. They included provisioning, deployment, new hypervisors including increased performance, elasticity and scheduling, resource management, benchmarking, emulation, and IaaS scaling to support MOOC’s. 4 projects studied distributed clouds and storage with federation. 3 projects centered on networking (routing, optimized devices and emulation) and 2 studied software engineering for clouds. 5 projects were aimed at cyberphysical systems with health, power and mobile applications and FutureGrid used for control and support analytics. 2 projects had a P2P focus covering security and fault tolerance. Security was popular with 10 projects covering trusted P2P storage, mobile and other clients, intrusion detection, file sharing, confidentiality and integrity of data, vulnerability, watermarks and hybrid clouds with different security models. There were 4 HPC programming language projects and 8 covering cloud programming with scheduling, fault tolerance and runtime. 5 fault tolerance projects covered P2P, MapReduce, workflow and HPC. 4 projects covered distributed software transactional memory and concurrency control. Data systems were very popular with 13 projects covering cloud storage, data transfer, NoSQL, streaming big data, Apache software stack, analytics, image retrieval, testing, provenance and the semantic web. 2 projects studied enterprise software issues. 9 artificial intelligence projects covered learning networks for images and social media, large scale image classification by clustering, machine learning, text mining, and agents. Network science with 3 projects saw use of NoSQL datastores to study Twitter, a major infrastructure to support graph and other tools and community detection. Finally 13 projects focused on cyberinfrastructure middleware with XSEDE and EMI (European Middleware initiative) systems, software as a service for HPC simulations, HPC clouds, MPI porting, fault tolerance, workflow, scheduling and resource management, tools, logging and performance.
There were 19 Evaluation projects which covered XSEDE testing (a major actual and intended use of FutureGrid), Open Science Grid testing, particle physics, Apache Big Data stack, Solid State disks, comparison of different VM frameworks, familiarization with cloud technology, GPU’s and studies aimed at planning institutional initiatives in cloud computing. There were 3 Interoperability projects involving long term support of standard end-points. 

FutureGrid interacts with XSEDE on integrating accounting approaches, EOT and software testing.


\end{document}
