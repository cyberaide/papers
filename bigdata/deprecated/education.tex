

\subsection{Support for Education}

To support the many educational and research projects on FutureGrid we have provided through a portal significant amount of material on how to use the discussed services. In addition, we realized that not every educational project has users with sophisticated computer usage and we hance provide for such projects the ability to a streamlined user interface arther than having the users fight with complex command line syntax and parameters. Thus we provided for a recent MOOC on big data taught with resources on FutureGrid the basic funtionality not only to start VMs as part of the IaaS framework, but also to deploy sophisticated images that contain preinstalled software and allow sorvices to be hosted by the users such as iPypthon, R and much more. This was implemented on top of OpenSTack while utilizing the newest OpenStack services such as heat. The management of the VMs and starting of the iPython server was controlled by a python application that provides the user with a menu system. Thus the management of them becam literraly, as easy as pressing 1, 2, 3, ... in the menu.
For other classes we also have provided completely separate OpenStack deployments as the teachers were afraid that students woudl not have enough resources. However we learned from this that the teachers overestimated the actual utilization of the project and many resources were not used. Based on this analysis we do now have a model to justify the creation more relaxed access policies and potentially justify that even classes shoudl be provided in the main region that FG provides. If resource contention would become an issue we coudl set aside a special region for a limited number of time.
Reconfiguration needs have alo arisen where one day a class may want o explore traditional MPI, while the next they want to experiment with Hadoop.
Furthermore, we identified that several users wanted to combine various cloud IaaS platforms in order to avoid resource overprovisioning, or were interested in combining all resources. 
