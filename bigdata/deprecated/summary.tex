\section{Summary}

In this chapter we have described FutureGrid and focused on services
that are beneficial for big data analysis. Based on the discussion it
is clear that such a system is extremly complex but provides many
benefits of offering multiple services within the same
infrastructure. Performance experiments can thus be not only conducted
while conducting big data analysis in virtual amchines, but on a
variety of IaaS and PaaS enviroenments. Moreoverthes experiemnets can
directly be compared to bare metal provisioned services. Hence, users
can evaluate what impact such technologies have on their
codes. Comparisions of different progarmming farmeworks can be
achieved and future activities in regards to efficiency and usability
can be deducted. The lessons learned from FutureGrid are motivating a
toolkit cloudmesh that already today allows to manage virtual machines
on a avriety of infrastructure as a service frameworks. The easy
deployment of sophisticated setups with a one click deployment has
been validated as part of an infrastructure designed for a
MOOC. Furthermore the novel concept of shifting resources
\cite{las08federated-cloud} between
services to topport services that need more resources is a significant
contribution by cloudmesh. Image management and creation under
security restrictions \cite{fg-1295}  is
furthermore an important aspect. We will continue to develop the
cloudmesh environment and make it available to our users. 