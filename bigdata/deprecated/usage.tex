\FILE{usage.tex}

\section{FutureGrid Usage}

When offering services such as FutureGrid to the community, we have to
analyse and predict which services may be useful for the users. We
have therefore established a number of activities that monitor
external and internal data. Externally, we look for example at
information provided by Gartners technology hype curve \cite{?} or
Google trend information as shown in Figure \ref{F:google-trend}. From
Google Trend data we observe that the popularity of Grid computing
has been very low  in the recent years and much attention has shifted
to cloud computing. Therefore we removed this information from the
figure and focus exemplary on cloud related terms such as {\em Cloud
  Computing}, {\em Big Data}, {\em OpenStack} {\em VMWare}.
From this information we see that all but VMWare are rising, with
Cloud Computing dominating the google trends in comparison to the
others. This trend is important as it shows a shift in the cloud
computing community buzz away from a traditional commercial market
leader in virtualization technology. We believe that is correlated
with a large number of vendors offering alternative products and
services while at the same time the novelty from VMWare is reduced.

\begin{figure}[htb]
 \centering
    \includegraphics[width=.75\textwidth]{images/google-trend.pdf}
  \caption{Google Trends.}\label{F:google-trend}
\end{figure}

To give an informal overview of the more than 300 projects conducted
on FutureGrid we have taken their titles and display them in a word
cloud (see Figure \ref{F:wordcloud}. Additionally, we have taken
keywords that are provided by the project leads and also displayed
the in a word cloud (see Figure \ref{F:keycloud}. Although the
images doe not give quantitative perspective about the project it helps
to identify some rough idea about the activities that are ongoing in FutureGrid.
As expected the terms cloud computing an and terms such as mapreduce,
Openstack, Nimbus, and Eucalyptus appear quite frequently. It is hence
worthwhile to analyse this data in a more quantitative form.

\begin{figure}[p]
\begin{minipage}[t]{1.0\textwidth}
  \centering
    \includegraphics[width=1.0\textwidth]{images/fg-title-wordcloud.pdf}
  \caption{Project title word cloud.}\label{F:wordcloud}
\end{minipage}
\vspace{24pt}\\
\begin{minipage}[t]{1.0\textwidth}
  \centering
    \includegraphics[width=1.0\textwidth]{images/fg-keyword-wordcloud.pdf}
  \caption{Project keyword word cloud.}\label{F:keycloud}
\end{minipage}
\end{figure}


As part of our project management in FutureGird we have designed a
simple project application procedure that includes prior to a project
granted access gathering information about which technologies are
anticipated to be used within the project. The list is fairly
extensive and includes Grid, HPC, and Cloud computing systems,
services, and software. However, for this paper we will focus
primarily on technologies that are dominantly requested and depicted
in Figure~\ref{F:request-tech}. Clearly we can identify the trend,
that shows the increased popularity of OpenStack within the services
offered on FutureGrid. Nimbus and Eucalyptus are on a significant
downwards trend. ObenNebula was also at one point more requested that
either Nimbus or Eucalyptus, but do to limited manpower an official
version of OpenNebula was not made available. As we have not offered
it and pointed it out on our Web page, requests for OpenNebula have vanished.
However internally have used OpenNebula for projects such as our cloudmesh rain
framework. All other sixteen technologies are relatively equally
distributed over the monitoring period. The lesson that we took form
this is that FutureGrid has put more emphasize in offering OpenStack Services.

\begin{figure}[htb]
  \centering
    \includegraphics[width=1.0\textwidth]{images/trend-a.pdf}
  \caption{Requested technologies by project}\label{F:request-tech}
\end{figure}

\begin{figure}[p]
 \centering
    \includegraphics[width=.7\textwidth]{images/project-frequency.pdf}
  \caption{Project Frequency.}\label{F:project-members}

  \centering
    \includegraphics[width=.6\textwidth]{images/project-disciplines.pdf}
  \caption{Distributon of project disciplines.}
  \label{F:freq-dis}

  \centering
    \includegraphics[width=.75\textwidth]{images/trend-b.pdf}
  \caption{Requests by of technologies by discipline within a
    project. $\bigtriangleup$ = Map Reduce, Hadoop, or Twister,
    $\Box$  = MPI, $\circ$ = ScaleMP}
   \label{F:trend-b}
\end{figure}

From the overall project information we have also analysed the
frequency of the number of project members within the project and show
it in Figure~\ref{F:project-members}. Here we depict on the abscissa
classes of projects with varying members.  Assume we look at the
abscissa value of 10, This means that these are all projects that
have project members between 10 and its previous category in this case
5. Hence, it will be all projects greater  5 and smaller or equal
10. With this classification  we see that the dominant unique number of
members within all projects is either one, two or there members. Than
we have another class between 4 and 10 members, and the rest with more
than ten members. One of the projects had overall 186 registered
members, for an education class as part of a summer school. Looking at
the distribution of the members and associating them with research and
education projects, we find all projects with larger numbers of
projects to be education projects.



%Figure~\ref{F:bigdata-freq} shows the frequency distribution of technologies such as
%map/reduce, hadoop, and twister by year. In the chart we simply called
%the agglomoration of these technologies big data.  As we see relative
%to all projects requested within one year we identified a rising trend.

%\begin{figure}[htb]
%  \centering
%    \includegraphics[width=0.2\textwidth]{images/bigdata-freq.pdf}
%  \caption{Big Data Project frequency.}\label{F:bigdata-freq}
%\end{figure}

Next we have analyzed the for all projects that requested either mapreduce,
hadoop, twister, MPI and ScaleMP (147 of all 374 active projects,
which is 39\% of all projects) and categorized them by discipline as
shown in in \ref{F:freq-dis}. In contrast to XSEDE which provides a production HPC
system to the scientific community, the usage of FutureGrid is
dominated with 50\% by computer science related projects\footnote{is
  this for all projects or just within hadoop? This is not cear from
  the data I received.} Education is the next highest with 19\%.


If we look at further into this data we present in Figure
\ref{F:trend-b} The number of projects in a particular category, as
well as the Fraction of technologies within a discipline. As we are in
this paper interested in the impact on big data. we have looked in
particular at requests for mapreduce, hadoop, and twister, awhile also looking at
requests for MPI and ScaleMP. It is interesting to note that the
percentual distribution of the technologies among these projects is
about constant if we exclude technology evaluations and interoperability. As
MPI is more popular with domain sciences we find a slight increase in
projects requesting MPI. However with the life sciences we see the
oposite as mapreduce and associated technologies are more popular
here. MPI and ScaleMP are not much requested as part of technology
evaluations and interoperability experimentation is as they either
project a very stable framework and does not require evaluation, or
the question of interoperability is not of concern for most of the projects.
\footnote{slash box does not show up, do we need another latex package}




%\begin{figure}[htb]
%  \centering
%    \includegraphics[width=1.0\textwidth]{images/project-member-dist.pdf}
%  \caption{Project Member Dist.}
%\end{figure}

%\afterpage{\clearpage}