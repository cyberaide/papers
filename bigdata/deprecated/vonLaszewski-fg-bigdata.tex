%% LyX 2.0.6 created this file.  For more info, see http://www.lyx.org/.
%% Do not edit unless you really know what you are doing.
\documentclass[english]{article}
\usepackage[T1]{fontenc}
\usepackage[latin9]{inputenc}
\usepackage{textcomp}

\makeatletter
%%%%%%%%%%%%%%%%%%%%%%%%%%%%%% Textclass specific LaTeX commands.
\newcommand{\lyxaddress}[1]{
\par {\raggedright #1
\vspace{1.4em}
\noindent\par}
}
\newenvironment{lyxcode}
{\par\begin{list}{}{
\setlength{\rightmargin}{\leftmargin}
\setlength{\listparindent}{0pt}% needed for AMS classes
\raggedright
\setlength{\itemsep}{0pt}
\setlength{\parsep}{0pt}
\normalfont\ttfamily}%
 \item[]}
{\end{list}}

\makeatother

\usepackage{babel}
\begin{document}

\title{FutureGrid testbed for Big Data }


\author{Gregor von Laszewsi, Geoffrey C. Fox }

\maketitle

\lyxaddress{laszewski@gmail.com}
\begin{abstract}
FutureGrid testbed for Big Data Gregor von Laszewsi, Geoffrey C. Fox
laszewski@gmail.com

Abstract In this chapter we will be introducing you to FutureGrid
that provides a testbed to conduct research for Cloud, Grid, and High
Performance Computing. Although FutureGrid has only a relatively small
number of compute cores (about 4500 regu-lar cores and 14000 GPU cores)
it provides an ideal playground to test out various frameworks that
may be useful for users to consider as part of their big data analy-sis
pipelines.

The chapter is structured as follows. First we will provide the reader
with an in-troduction to Future Grid. We will list a number of projects
that use Futuregrid to conduct data analysis and introduce some of
them to the reader. We will tell you about which services and Hardware
exists. Next we will analyze which services are preinstalled and are
available for big data analysis. As services that users may need for
their work we point out how such a testbed can be utilized not only
while provisioning virtual machines, but also on bare metal.\end{abstract}
\begin{lyxcode}
We~conclude~the~chapter~with~our~observation~cast~throught~three~years~of~op-erating~FutureGrid~and~provide~an~outlook~for~the~next~steps.
\end{lyxcode}

\section{Introduction }
\begin{abstract}
FutureGrid {[}FG{]} \textquotedblleft{}is a project led by Indiana
University and funded by the Na-tional Science Foundation (NSF) to
develop a high-performance grid test bed that will allow scientists
to collaboratively develop and test innovative approaches to parallel,
grid, and cloud computing. FutureGrid will provide the infrastructure
to researchers that allows them to perform their own computational
experiments us-ing distributed systems. The goal is to make it easier
for scientists to conduct such experiments in a transparent manner.
FutureGrid users will be able to deploy their own hardware and software
con-figurations on a public/private cloud, and run their experiments.
They will be able to save their configurations and execute their experiments
using the provided tools. The FutureGrid test bed is composed of a
high-speed network connecting distributed clusters of high performance
computers. FutureGrid employs virtual-ization technology that will
allow the test bed to support a wide range of operating systems.\textquotedblright{}

2.Overview of FutureGrid for Big Data 2.1 Service Overview

According to the manual FutureGrid provides a number of different
services. These services include: OpenStack which includes a collection
of open source components to deliver public and private clouds. These
components currently include OpenStack Com-pute) OpenStack Object
Storage, and OpenStack Image Service. OpenStack has received considerable
momentum due to its openness and the support of compa-nies. Nimbus
which is an open-source service package that allows users to run vir-tual
machines on FutureGrid hardware. Just as in Openstack users can upload
their own virtual machine images or customize existing once. Nimbusnext
to Eucalyp-tus is one of the earlier frameworks that make managing
virtual machines easier. Eucalyptus is an open-source software platform
that implements IaaS-style cloud computing. Eucalyptus provides an
Amazon Web Services (AWS) compli-ant EC2-based web service interface
for interacting with the Cloud service. Euca-lyptus has been previously
the dominant alternative to AWS in academia. How-ever, based on usage
patterns in FutureGrid we believe it is replaced by OpenStack. High
Performance Computing can be defined as the application of super-computing
techniques to solve computational problems that are too large for
standard computers or would take too much time. This is one of the
more im-portant features that the scientific community needs to achieve
their projects. Nat-urally using HPC resouces and services is also
useful in the area of Big Data. Sometimes big data needs big machines.
Thus, using HPC may be an ovious choice. Map Reduce \ldots{}. TBD
\ldots{} Storage on FutureGrid has moderate size storage capability
that will satisfy the users demand to compare and test someof the
previously outlined services. Information Services gather the information
of the different elements that make up FutureGrid to provide accurate
and complete knowledge of the computa-tional environment. This information
is presented using different web portals.

2.2Hardware Overview According to the manual, FutureGrid is build
out of a number of clusters of dif-ferent type and size that are interconected
with up to a 10GB Ethernet among its sites. The sites include Indiana
University, University of Chicago, San Diego Su-percomputing Center,
Texas Advanced Computing Center, and University of Flor-ida.

Overview of the Clusters

NameSystem Type\# Nodes\# CPUS\# CoresTFLOPSRAM (GB)Storage (TB)Site
indiaIBM iDataplex1282561024113072335IU hotelIBM iDataplex8416867272016120UC
sierraIBM iDataplex841686727268896SDSC foxtrotIBM iDataplex326425637680UF
alamoDell Poweredge961927688115230TACC xrayCray XT5m1166664613285.4IU
bravoHP Proliant16321281.73072128IU deltaSuperMicro GPU Cluster16321921333144IU
limaAeon Eclipse648161281.35123.8SDSC echoSuperMicro ScaleMP Cluster163219226144192IU

3. Services and Tools for Big Data 3.1 High Performance Computing

3.2 Hadoop My Hadoop

Cloudmesh

Acknowledgement Some of the text published in this chapter is available
form the FutureGrid por-tal. The FutureGrid project is funded by the
National Science Foundation (NSF) and is led by Indiana University
with University of Chicago, University of Florida, San Diego Supercomputing
Center, Texas Advanced Computing Center, Universi-ty of Virginia,
University of Tennessee, University of Southern California, Dres-den,
Purdue University, and Grid 5000 as partner sites. This material is
based up-on work supported in part by the National Science Foundation
under Grant No. 0910812.

If you use FutureGrid, we ask you to include the following reference
in your papers the following reference in addition to this chapter:
Fox, G., G. von Laszewski, J. Diaz, K. Keahey, J. Fortes, R. Figueiredo,
S. Smallen, W. Smith, and A. Grimshaw, \textquotedblleft{}FutureGrid
- a reconfigurable testbed for Cloud, HPC and Grid Computing\textquotedblright{},
Contemporary High Performance Computing: From Petascale to-ward Exascale,
April, 2013. Editor J. Vetter. Contemporary High Performance Computing:
From Petascale toward Exascale, April, 2013.

References \textbullet{}{[}FG{]} Fox, G., G. von Laszewski, J. Diaz,
K. Keahey, J. Fortes, R. Figueiredo, S. Smallen, W. Smith, and A.
Grimshaw, \textquotedblleft{}FutureGrid - a reconfigurable testbed
for Cloud, HPC and Grid Computing\textquotedblright{}, Con-temporary
High Performance Computing: From Petascale toward Exascale, April,
2013. Editor J. Vetter. Contemporary High Perfor-mance Computing:
From Petascale toward Exascale, April, 2013.\end{abstract}

\end{document}
