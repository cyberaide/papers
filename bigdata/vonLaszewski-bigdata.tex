\documentclass{article}

%%%%%%%%%%%%%%%%%%%%%%%%%%%%%%%%%%%%%%%%%%%%%%%%%%%%%%%%%%%%%%%%%%%%%%
% LATEX DEFINITIONS
%%%%%%%%%%%%%%%%%%%%%%%%%%%%%%%%%%%%%%%%%%%%%%%%%%%%%%%%%%%%%%%%%%%%%%

\usepackage{array}
\usepackage{graphicx}
\usepackage{booktabs}
\usepackage{pifont}
\usepackage{todonotes}
\usepackage{rotating}
\usepackage{color}

\newcommand*\rot{\rotatebox{90}}

\newcommand{\FILE}[1]{\todo[color=green!40]{#1}}

%%%%%%%%%%%%%%%%%%%%%%%%%%%%%%%%%%%%%%%%%%%%%%%%%%%%%%%%%%%%%%%%%%%%%%
% TITLE ABLE OF CONTENTS
%%%%%%%%%%%%%%%%%%%%%%%%%%%%%%%%%%%%%%%%%%%%%%%%%%%%%%%%%%%%%%%%%%%%%%

\newcommand{\TITLE}{The FutureGrid Testbed for Big Data}
\newcommand{\AUTHOR}{Gregor von Laszewsi, Geoffrey C. Fox}
\newcommand{\EMAIL}{laszewski@gmail.com}
\begin{document}

\title{\TITLE}
\author{\AUTHOR}
\date{\EMAIL}


%%%%%%%%%%%%%%%%%%%%%%%%%%%%%%%%%%%%%%%%%%%%%%%%%%%%%%%%%%%%%%%%%%%%%%
% TABLE OF CONTENTS
%%%%%%%%%%%%%%%%%%%%%%%%%%%%%%%%%%%%%%%%%%%%%%%%%%%%%%%%%%%%%%%%%%%%%%

\pagenumbering{roman}

\begin{center}
{\Large\bf \TITLE}\\
{\AUTHOR}\\
{\EMAIL}
\end{center}

\tableofcontents

\newpage

%%%%%%%%%%%%%%%%%%%%%%%%%%%%%%%%%%%%%%%%%%%%%%%%%%%%%%%%%%%%%%%%%%%%%%
% LIST OF TODOS
%%%%%%%%%%%%%%%%%%%%%%%%%%%%%%%%%%%%%%%%%%%%%%%%%%%%%%%%%%%%%%%%%%%%%%

\listoftodos

\newpage

%%%%%%%%%%%%%%%%%%%%%%%%%%%%%%%%%%%%%%%%%%%%%%%%%%%%%%%%%%%%%%%%%%%%%%
% TITLE OF PAPER
%%%%%%%%%%%%%%%%%%%%%%%%%%%%%%%%%%%%%%%%%%%%%%%%%%%%%%%%%%%%%%%%%%%%%%

\pagenumbering{arabic}

\maketitle

%%%%%%%%%%%%%%%%%%%%%%%%%%%%%%%%%%%%%%%%%%%%%%%%%%%%%%%%%%%%%%%%%%%%%%
% ABSTRACT OF PAPER
%%%%%%%%%%%%%%%%%%%%%%%%%%%%%%%%%%%%%%%%%%%%%%%%%%%%%%%%%%%%%%%%%%%%%%


\begin{abstract}

In this chapter we will be introducing you to FutureGrid that provides a testbed to conduct research for Cloud, Grid, and High Performance Computing. Although FutureGrid has only a modest number of compute cores (about 4500 regular cores and 14000 GPU cores) it provides an ideal playground to test out various frameworks that may be useful for users to consider as part of their big data analysis pipelines. 

The chapter is structured as follows. First we will provide the reader with an introduction to Future Grid. We will list a number of projects that use Futuregrid to conduct data analysis and introduce some of them to the reader. We will tell you about which services and Hardware exists. Next we will analyze which services are preinstalled and are available for big data analysis. As services that users may need for their work we point out how such a testbed can be utilized not only while provisioning virtual machines, but also on bare metal. 

We conclude the chapter with our observation cast throught three years of operating FutureGrid and provide an outlook for the next steps.
\end{abstract}

%%%%%%%%%%%%%%%%%%%%%%%%%%%%%%%%%%%%%%%%%%%%%%%%%%%%%%%%%%%%%%%%%%%%%%
% SECTIONS
%%%%%%%%%%%%%%%%%%%%%%%%%%%%%%%%%%%%%%%%%%%%%%%%%%%%%%%%%%%%%%%%%%%%%%

\section{Introduction}

FutureGrid \cite{las2010gce,las12fg-bookchapter} ``is a project led by Indiana University and funded by the National Science Foundation (NSF) to develop a highperformance grid test bed that will allow scientists to collaboratively develop and test innovative approaches to parallel, grid, and cloud computing. FutureGrid will provide the infrastructure to researchers that allows them to perform their own computational experiments using distributed systems. The goal is to make it easier for scientists to conduct such experiments in a transparent manner.  FutureGrid users will be able to deploy their own hardware and software configurations on a public/private cloud, and run their experiments. They will be able to save their configurations and execute their experiments using the provided tools. The FutureGrid test bed is composed of a highspeed network connecting distributed clusters of high performance computers. FutureGrid employs virtualization technology that will allow the test bed to support a wide range of operating systems.''



\section{Overview of FutureGrid}

\subsection{Hardware Overview}

According to the manual, FutureGrid is build out of a number of clusters of different type and size that are interconected with up to a 10GB Ethernet among its sites. The sites include Indiana University, University of Chicago, San Diego Supercomputing Center, Texas Advanced Computing Center, and University of Florida.

\subsubsection{Overview of the Clusters}

\begin{table}[htb]

\caption{FutureGrid Compute Resources}\label{T:hw}

\begin{center}
\begin{tabular}{rrrrrrrrr}
Name    & System Type                &  \rot{Nodes} &  \rot{CPUS}   & \rot{Cores}   & \rot{TFLOPS}  & \rot{RAM (GB)}        & \rot{Storage (TB)}    & Site \\
\hline
india   & IBM iDataplex              & 128          & 256     & 1024    & 11      & 3072            & 335             & IU \\
hotel   & IBM iDataplex              & 84           & 168     & 672     & 7       & 2016            & 120             & UC \\
sierra  & IBM iDataplex              & 84           & 168     & 672     & 7       & 2688            & 96              & SDSC \\
foxtrot & IBM iDataplex              & 32           & 64      & 256     & 3       & 768             & 0               & UF \\
alamo   & Dell Poweredge             & 96           & 192     & 768     & 8       & 1152            & 30              & TACC \\
xray    & Cray XT5m                  & 1            & 166     & 664     & 6       & 1328            & 5.4             & IU \\
bravo   & HP Proliant                & 16           & 32      & 128     & 1.7     & 3072            & 128             & IU \\
delta   & \shortstack{SuperMicro\\ GPU Cluster}     & 16           & 32      & 192     &         & 1333            & 144             & IU \\
lima    & Aeon Eclipse64             & 8            & 16      & 128     & 1.3     & 512             & 3.8             & SDSC \\
echo    & \shortstack{SuperMicro \\ScaleMP Cluster} & 16           & 32      & 192     & 2       & 6144            & 192             & IU \\
\end{tabular}
\end{center}
\end{table}


\FILE{hw-table.tex}

\begin{sidewaystable}

\caption{FutureGrid cluster details.}\label{F:cluster-details}
~\\
\begin{footnotesize}
\begin{tabular}{|p{2cm}||p{4cm}p{1.5cm}p{1.5cm}p{1.5cm}p{1.5cm}p{1.5cm}p{1.5cm}p{1.5cm}p{1cm}|}
\hline
 \bf Name                                & \bf Echo & \bf Alamo & \bf Bravo & \bf Delta & \bf Foxtrot & \bf Hotel & \bf India & \bf Sierra & \bf Xray \\
\hline
\hline
 Organization                        & IU & TACC & IU & IU & UF & UC & IU & SDSC & IU \\
\hline
 Machine Type                        & Cluster SclaeMP & Cluster & Cluster & Cluster & Cluster & Cluster & Cluster & Cluster & Cluster \\
\hline
 System Type                         &SuperMicro& Dell PowerEdge M610 Blade & HP Proliant && IBM iDataPlex dx 360 M2 & IBM iDataPlex dx 360 M2 & IBM iDataPlex dx 360 M2 & IBM iDataPlex dx 340 & Cray XT5m \\
\hline
 CPU Type                            & Xeon E5-2640 &  Xeon X5550 &  Xeon E5620 &  Xeon 5660 &  Xeon X5520 &  Xeon X5550 &  Xeon X5550 &  Xeon L5420 & Opteron 2378 \\
\hline
 CPU Speed                           &2.50GHz& 2.66GHz & 2.40GHz & 2.80 GHz & 2.26GHz & 2.66GHz & 2.66GHz & 2.5GHz & 2.4GHz \\
\hline
 CPUs                                &&192&32&32&64&168&256&168&168 \\
\hline
 Servers                             &12&96&16&16&32&84&128&84&1 \\
\hline
 RAM                                 && 12GB DDR3 1333Mhz & 192GB DDR3 1333Mhz & 192GB DDR3 1333 Mhz & 24GB DDR3 1333Mhz & 24GB DDR3 1333Mhz & 24GB DDR3 1333Mhz & 32GB DDR2-667 & 8GB DDR2-800 \\
\hline
 Total RAM                           &&1152GB&3072GB&3072GB&768GB&2016GB&3072GB&2688GB&1344GB \\
\hline
 Number of cores                     &144&768&128&&256&672&1024&672&672 \\
\hline
 Tflops                              &&8&1.7&&3&7&11&7&6 \\
\hline
 Disk Size (TB)                      &2.8&48&&15&20&120&335&72&335 \\
\hline
 Hard Drives                         && 500GB 7.2K RPM SAS & 6x2TB 7.2K RPM SATA & 92GB 7.2K RPM SAS2 & 500GB 7200 RPM SATA & 1 TB 7200 RPM SATA & 3000GB 7200 RPM SATA & 160GB 7200 RPM SATA Drive & 6TB Lustre \\
\hline
 Shared Storage                      && NFS & NFS &NFS& NFS & GPFS & NFS & ZFS 82.2TB & NFS \\
\hline
 Interconnect                        && Mellanox 4x QDR IB & Mellanox 4x DDR IB &&& Mellanox 4x DDR IB & Mellanox 4x DDR IB & Mellanox 4x DDR IB & Cray SeaStar \\
\hline
\end{tabular}
~\\
IB = InfiniBand, Xenon = INtel Xenon, Opteron = AMD Opteron 

\end{footnotesize}

\end{sidewaystable}

\subsubsection{Overview of Networking}

\subsubsection{Overview of Storage}



\todo{services.tex}
\section{Services and Tools for Big Data}

\subsection{Traditional High Performance Computing}

The traditional high performance computing environment provied by queuing systems and Message Passing Interface (MPI) programs provide a suitable infrastructure not only for simulations, but also for the analysis of large data. However, considerable amount of work has to be put in place to optimize the available infarstructure for the problem doamin. This has been successfully demonstrated for many biological applications. Additinally the existance of a queuing system can provide some advantages when the available resources are utilized to a full extend and resource starvation exists while sharing the resources with other users. This has been esppecially useful to also support educational activities for clases with many users that for example want to test map reduce activities controled by a queing system as described in Section \ref{S:hadoop}.

\subsection{Virtual Large-Memory System}

One of the demands often posed in big data analysis it to place the data as much as possible into memory to speed up calculations and in some cases to fit the entire dataset into memore. However, this analysis may come at a cost as for example the use of HPC computing via MPI adds additional programming complexity within a cluster. Therefore it is desirable to virtualize the memory from multiple servers in a cluster to provide one big memory system that can be easily accessed by the underlying software. 
One such implementation, vSMP by ScaleMP \cite{www-scalemp}.
Experiments conducted on futureGrid using HPCC
benchmarks show only a 4-6\% drop in efficiency when compared to native
cluster performance \cite{las12fg-bookchapter}. This makes it feasable for many applications. ScaleMP is installed on the FutureGrid echo cluster that has 16 servers and can access up to 3TB in shared virtual memory.

\subsection{Infrastructure as a Service}



According to the manual FutureGrid provides a number of different
services. These services include:

\begin{enumerate}
\item OpenStack which includes a collection of open source components to deliver public and private clouds. These components currently include OpenStack Compute) OpenStack Object Storage, and OpenStack Image Service. OpenStack has received considerable momentum due to its openness and the support of companies. 

\item Nimbus which is an opensource service package that allows users to run virtual machines on FutureGrid hardware. Just as in Openstack users can upload their own virtual machine images or customize existing once. Nimbusnext to Eucalyptus is one of the earlier frameworks that make managing virtual machines easier.

\item Eucalyptus is an opensource software platform that implements IaaS-style cloud computing. Eucalyptus provides an Amazon Web Services (AWS) compliant EC2-based web service interface for interacting with the Cloud service.  Eucalyptus has been previously the dominant alternative to AWS  in academia. However, based on usage patterns in FutureGrid we believe it is replaced by OpenStack.

\item High Performance Computing can be defined as the application of supercomputing techniques to solve computational problems that are too large for standard computers or would take too much time. This is one of the more important features that the scientific community needs to achieve their projects. Naturally using HPC resouces and services is also useful in the area of Big Data. Sometimes big data needs big machines. Thus, using HPC may be an ovious choice.

\item Map Reduce …. TBD …

\end{enumerate}

Storage on FutureGrid has moderate size storage capability that will satisfy the users demand to compare and test someof the previously outlined services.

Information Services gather the information of the different elements that make up FutureGrid to provide accurate and complete knowledge of the computational environment. This information is presented using different web portals.

\begin{figure}[p]
  \centering
    \includegraphics[width=0.9\textwidth]{images/user-services.pdf}
  \caption{FutureGrid High Level User Services.}
  ~\\
  \centering
  \includegraphics[width=0.9\textwidth]{images/architecture.pdf}
  \caption{Architecture.}
\end{figure}

\section{Hadoop}



From http://hadoop.apache.org

''The Apache Hadoop project develops opensource software for reliable, scalable, distributed computing.

The Apache Hadoop software library is a framework that allows for the distributed processing of large data sets across clusters of computers using simple programming models. It is designed to scale up from single servers to thousands of machines, each offering local computation and storage. Rather than rely on hardware to deliver highavailability, the library itself is designed to detect and handle failures at the application layer, so delivering a highlyavailable service on top of a cluster of computers, each of which may be prone to failures.
'' \cite{www/hadoop}

from sriram \cite{report/myhadoop}

\section{Hadoop}

Traditional HPC environments typically support batch job submissions using resource
management systems such as the TORQUE Resource Manager (also known as the
Portable Batch System – PBS) or the Sun Grid Engine (SGE). On the other hand, Hadoop
provides it own scheduling, and manages its own job and task submissions, and tracking.
Since both systems are designed to have complete control over the resources that they
manage, the challenge is how to enable users to run Hadoop jobs in a typical HPC
environment using a scheduler such as PBS or SGE. In this release, we support Hadoop
job submissions via PBS and SGE. However, this approach is equally feasible for other
schedulers such as Condor, as well.
Our approach is to configure Hadoop clusters “on-demand” by first requesting resources
for an Nnode Hadoop cluster via PBS. Once the resources are received, the Hadoop
configurations and environments are set up based on the set of resources provided by
PBS. The Hadoop Distributed File System (HDFS) can be configured in one of two ways
– in 1) transient (nonpersistent) or 2) persistent modes. In the nonpersistent mode, the
HDFS is set up to use local storage. In the persistent mode, the HDFS is set to
symbolically link to an external location that will be persistent – i.e. data from Hadoop
runs will continue to persist even after the Hadoop runs are complete. More details are as
follows.

\subsection{Details}

The prerequisite for myHadoop is a valid Hadoop installation – we recommend that you
use Hadoop version 0.20.2 since that is the only version of Hadoop that this package has
been tested with. Henceforth, we will refer to the location of the Hadoop installation as
HADOOP\_HOME. We will refer to the location of the myHadoop installation (i.e. this
package) as MY\_HADOOP\_HOME. The \$MY\_HADOOP\_HOME/pbs-example.sh shows
an example of how to use myHadoop with PBS. A similar script for SGE can be found in
\$MY\_HADOOP\_HOME/sge-example.sh.
A step-by-step process for using myHadoop is as follows.

\subsection{Initial Configuration}

Ensure that the environment variables inside \$MY\_HADOOP\_HOME/bin/setenv.sh are
set correctly. You can set your HADOOP\_HOME, and the locations for your HDFS data
and log directories using this script. You will need to update this script before you can
proceed further.
All the tuning parameters for Hadoop can be found in the \$MY\_HADOOP\_HOME/etc
directory. There is no need to edit any of the parameters, especially if you are not an
expert Hadoop user. If you are familiar with the various Hadoop parameters, you may
edit the parameters that fall outside the “DO NOT EDIT” sections.

\subsection{Request N nodes from the Scheduler}

Once the environment variables have been set correctly, we are ready to use myHadoop
using a regular PBS or SGE submission script. Your PBS script should contain the
following lines to initialize PBS as follows:

\begin{verbatim}
#!/bin/bash
#PBS -q <queue_name>
#PBS -N <job_name>
#PBS -l nodes=4:ppn=1
#PBS -o <output file>
#PBS -e <error_file>
#PBS -A <allocation>
#PBS -V
#PBS -M <user email>
#PBS -m abe
\end{verbatim}

In the above case, we are requesting 4 nodes. Note that you must set the processors per
node (ppn) to 1.
Your SGE script should contain the following lines to initialize SGE:

\begin{verbatim}
#!/bin/bash
#$ -V -cwd
#$ -N <job_name>
#$ -pe <queue_name> 4
#$ -o <output file>
#$ -e <error file>
#$ -S /bin/bash
\end{verbatim}

For SGE, there is one important rule to remember. The queue name specified above
should be preconfigured with an allocation\_rule set to 1 (one). This ensures that the
Hadoop cluster is set up such that multiple instances of the Hadoop daemons are not
scheduled on the same node.

\subsection{Set the myHadoop Environment}

Run the \$MY\_HADOOP\_HOME/bin/setenv.sh script (that you modified in Section 2.1)
to set all the environment variables required by myHadoop.
. \$MY\_HADOOP\_HOME/bin/setenv.sh
Set the HADOOP\_CONF\_DIR to the directory where Hadoop configs should be
generated – all configuration files for the Hadoop run will be picked up from here. Ensure
that this directory is accessible to all nodes – and a way to do this is to make sure that this
directory is on a shared file system such as NFS or Lustre.
export HADOOP\_CONF\_DIR=<configuration directory>

\subsection{Configure the myHadoop Cluster}

You can initialize and configure the Hadoop cluster by using the
\$MY\_HADOOP\_HOME/bin/pbs-configure.sh (or sge-configure.sh) script. You may
create a transient or persistent myHadoop cluster by changing the commandline
arguments as follows.
For a transient myHadoop cluster, configure it as follows (replace 4 with the total number
of nodes requested):
\$MY\_HADOOP\_HOME/bin/pbs-configure.sh -n 4 -c \$HADOOP\_CONF\_DIR
In this mode, you will have to copy all of your data into the myHadoop cluster after it is
configured, and copy out the results after the job is complete. All data will be
inaccessible from HDFS once the PBS job is complete.
Alternatively, you may set up a persistent myHadoop cluster by using the –p option, and
setting the BASE\_DIR for HDFS as follows:
\$MY\_HADOOP\_HOME/bin/pbs-configure.sh -n 4 -c \$HADOOP\_CONF\_DIR -p -d
<HDFS BASE\_DIR>
The BASE\_DIR should be on a directory accessible to all nodes, to ensure that the data
will not be cleaned up after job completion. For instance, the BASE\_DIR could be on a
Lustre file system. Note that, if N-node cluster is being created, then the BASE\_DIR
should have directories named 1, 2, … , N. The configuration script sets up symbolic
links from node I to the BASE\_DIR/I directory. When this mode is used, there is no need
to copy data back and forth from HDFS to another file system between runs.

\subsection{Format HDFS (if need be)}

If myHadoop is being used in transient mode, or if it is being used for the first time in
persistent mode, then you will have to format the HDFS as follows:
\$HADOOP\_HOME/bin/hadoop --config \$HADOOP\_CONF\_DIR namenode –format

\subsection{Run Hadoop Jobs}

You are now all set to start all the Hadoop daemons as follows:
\$HADOOP\_HOME/bin/start-all.sh
Once the daemons are all started up, you can start using Hadoop as usual. You may also
stage data in and out from HDFS, as required.

\subsection{Clean up}

Although, PBS or SGE may be set up to automatically clean up after your Hadoop job is
complete, it is always a good idea to stop all the Hadoop daemons, and use the cleanup
script to clean up after yourself.
\$HADOOP\_HOME/bin/stop-all.sh
\$MY\_HADOOP\_HOME/bin/pbs-cleanup.sh -n 4 OR
\$MY\_HADOOP\_HOME/bin/sge-cleanup.sh -n 4


\subsection{Hadoop}

My Hadoop

We have various platforms that support Hadoop on FutureGrid. MyHadoop is probably the easiest solution offered for you. It provides the advantage that it is integrated into the queuing system and allows hadoop jobs to be run as batch job. This is of especial interest for classes that may run quickly out of resources if every students wants to run their hadoop application at the same time.



MapReduce is a programming model developed by Google. Their definition of MapReduce is as follows: “MapReduce is a programming model and an associated implementation for processing and generating large data sets. Users specify a map function that processes a key/value pair to generate a set of intermediate key/value pairs, and a reduce function that merges all intermediate values associated with the same intermediate key.” For more information about MapReduce, please see the Google paper here.

The Apache Hadoop Project provides an open source implementation of MapReduce and HDFS (Hadoop Distributed File System).

This tutorial illustrates how to run Apache Hadoop thru the batch systems on FutureGrid using the MyHadoop tool.

\subsubsection{myHadoop on FutureGrid}

MyHadoop is a set of scripts that configure and instantiate Hadoop as a batch job.

myHadoop 0.20.2 is currently installed on Alamo, Hotel, India, and Sierra FutureGrid systems.



\FILE{usage.tex}

\section{FutureGrid Usage}

\begin{figure}[htb]
  \centering
    \includegraphics[width=1.0\textwidth]{images/project-member-dist.pdf}
  \caption{Project Member Dist.}
\end{figure}

\begin{figure}[htb]
  \centering
    \includegraphics[width=1.0\textwidth]{images/google-trend.pdf}
  \caption{Google Trends.}
\end{figure}

\begin{figure}[htb]
  \centering
    \includegraphics[width=1.0\textwidth]{images/trend-a.pdf}
  \caption{Trend a.}
\end{figure}

\begin{figure}[htb]
  \centering
    \includegraphics[width=1.0\textwidth]{images/project-frequency.pdf}
  \caption{Project Frequency.}
\end{figure}

\begin{figure}[htb]
  \centering
    \includegraphics[width=1.0\textwidth]{images/trend-b.pdf}
  \caption{tend b.}
\end{figure}

\begin{figure}[htb]
  \centering
    \includegraphics[width=1.0\textwidth]{images/project-disciplines.pdf}
  \caption{project disciplines.}
\end{figure}



\FILE{devops.tex}

\section{System Management}

The goals of FutureGrid to offer a variety of services as part of its
testbed features is going beyond services that are normally offered by
data and supercomputing centers for research. This provides a number
of challenges that need to be overcome in order to efficiently manage
the system and provide services that have never been offered to users
as they exist on FutureGrid.

\subsection{Integration of Systems and Development Team}

FutureGrid started initially with a model where the systems team and
the software team were separated. A clear wall of responsibilities was
erected that resulted in multiple challenges:

\begin{enumerate}

\item
The system setup and management was completely separated from
the software development team focussing mostly on the deployment of
existing technologies. 

\item
The system was complex, but its deployment was documented to a limited
extend not allowing the developers to utilize it properly.

\item 
Lack of trust by the systems team dis not allow the software team to
have a valid development environment.

\item
The software developed needed a testbed within the
testbed that was not necessarily reflecting the actual system setup.

\end{enumerate}

Together these issues, made it extremely difficult if not impossible to
further any development in regards to the design of a testbed
infrastructure as requested by our original ambitious goals. 

To overcome these difficulties it was decided that the systems team
must be integrated in some fashion into the software team and become
part of the development process. This integration is not an isolated
instance within FutureGrid, but is also executed in many modern data
centers and is now recognized with its own term called {\em DevOps}.

\subsection{DevOps}

Devops is not just a buzzword from industry and research communities,
but it provides value added processes to the deployment and
development cycles that are part of modern data centers. It can today
be understood as a software development method that stresses
collaboration and integration between software developers and
information technology professionals such as a system administrator. 

WHile using an infrasrtructure such as clouds we recognized early on
that the lifetime of a particular IaaS framework is about 3-6 month
before a new version is installed. This is a significant difference to
a traditional High Performance Computing Center that is comprised of
many software tools experiencing much longer life spans. This is not
only based on security patches but significant changes for example in
the evolving security, user services as well as the deployment of new
services that become avalable in rapid procession.

This rapid change of the complex infrastructure requires a rethinking
about how systems in general are menaged and how they can be made
avalable to the development teams. While previously it may have been
enough to install updates on the machines, DevOps frameworks provide
the developer and system administarors to create and share
enviroenments that are used in production and development while at the
same time increasing quality assurance by leveraging each others
experiences (see Figure \ref{F:devops}).

\begin{figure}[htb]
  \centering
   \begin{minipage}{.5\textwidth}
    \includegraphics[width=1.0\textwidth]{images/devops.pdf}
    \caption{DevOps Intersection.}
    \label{F:devops}
  \end{minipage}%
   \begin{minipage}{.5\textwidth}
     \includegraphics[width=1.0\textwidth]{images/devops-circle.pdf}
     \caption{DevOps Cycle.}
     \label{F:devops-circel}
  \end{minipage}%
\end{figure}

\subsubsection{DevOps Cycle}

While combining the steps executed by the Development and operational
team from planing, to coding, building and testing, to the release,
deployment and operation and monitoring (see Figure
\ref{F:devops-circle}), each of the phases provides a direct feedback
between the DevOps team members and shortening thus the entire
development phase. It also allows to test out new services and
technologies in a rapid progression. Hence it is possible to roll out
new developments much faster into production. This leads to a much mor
rapid integrated cycle than without the correlation between
development and operation would be possible.

\subsubsection{DevOps Supporting Tools}

A number of tools are available that make the introduction of DevOps
strategies more efficient. The first is the need for an efficient
communication pathway to manage tasks not only between developers but
also between users. Thus the ideal system would provide a complete
integration of a project management system that allows to manage tasks
for both developers and operators, but also to easily integrate
tickets and transform them into tasks. In XSEDE and other
supercomputing centers a system called RT is typically used for user
ticket management. Other systems such as jira, mantis, and ??? are
often used to manage the software and systems related
tasks. Unfortunately, personal or organizational constraints prevent
often the integration of the two systems and additional overhead is
needed to move user tickets into tasks and the development
cycle. Within FutureGrid we experimented as part of our opensource
development extensively with jira as systems and ticketing system
reveling that newest development in such areas motivated by DevOps
teams lead to tools that support the overall cycle including users
(see Figure \ref{F:usedevops}). However, the integration of
FutureGrid within the overall much larger XSEDE effort did make it not
possible to switch from RT to jira for user ticket
management. To stress this user integration we term this framework
{\em UseDevOps}. Tools to integrate Development and Operation
deployment include puppet, chef, ansible, cfengine and bcfg2. While
FutureGrid started out with bcfg2 we have since than switched to other
tools due to their prevalence within the community. Chef, puppet, and
ansible have significant amount of traction. Due to expertise within
our group we currently explore chef and ansible. 

\begin{figure}[htb]
  \centering
    \includegraphics[width=0.5\textwidth]{images/usedevops.pdf}
  \caption{User Support integrated into DevOps leads to UseDevops.}
  \label{F:usedevops}
\end{figure}








%http://en.wikipedia.org/wiki/DevOps







\section{Services and Tools for Big Data}

\subsection{Support for Educational Services}

manual

reprovisioning

reconfiguration

\subsection{High Performance Computing}

\section{Hadoop}



From http://hadoop.apache.org

''The Apache Hadoop project develops opensource software for reliable, scalable, distributed computing.

The Apache Hadoop software library is a framework that allows for the distributed processing of large data sets across clusters of computers using simple programming models. It is designed to scale up from single servers to thousands of machines, each offering local computation and storage. Rather than rely on hardware to deliver highavailability, the library itself is designed to detect and handle failures at the application layer, so delivering a highlyavailable service on top of a cluster of computers, each of which may be prone to failures.
'' \cite{www/hadoop}

from sriram \cite{report/myhadoop}

\section{Hadoop}

Traditional HPC environments typically support batch job submissions using resource
management systems such as the TORQUE Resource Manager (also known as the
Portable Batch System – PBS) or the Sun Grid Engine (SGE). On the other hand, Hadoop
provides it own scheduling, and manages its own job and task submissions, and tracking.
Since both systems are designed to have complete control over the resources that they
manage, the challenge is how to enable users to run Hadoop jobs in a typical HPC
environment using a scheduler such as PBS or SGE. In this release, we support Hadoop
job submissions via PBS and SGE. However, this approach is equally feasible for other
schedulers such as Condor, as well.
Our approach is to configure Hadoop clusters “on-demand” by first requesting resources
for an Nnode Hadoop cluster via PBS. Once the resources are received, the Hadoop
configurations and environments are set up based on the set of resources provided by
PBS. The Hadoop Distributed File System (HDFS) can be configured in one of two ways
– in 1) transient (nonpersistent) or 2) persistent modes. In the nonpersistent mode, the
HDFS is set up to use local storage. In the persistent mode, the HDFS is set to
symbolically link to an external location that will be persistent – i.e. data from Hadoop
runs will continue to persist even after the Hadoop runs are complete. More details are as
follows.

\subsection{Details}

The prerequisite for myHadoop is a valid Hadoop installation – we recommend that you
use Hadoop version 0.20.2 since that is the only version of Hadoop that this package has
been tested with. Henceforth, we will refer to the location of the Hadoop installation as
HADOOP\_HOME. We will refer to the location of the myHadoop installation (i.e. this
package) as MY\_HADOOP\_HOME. The \$MY\_HADOOP\_HOME/pbs-example.sh shows
an example of how to use myHadoop with PBS. A similar script for SGE can be found in
\$MY\_HADOOP\_HOME/sge-example.sh.
A step-by-step process for using myHadoop is as follows.

\subsection{Initial Configuration}

Ensure that the environment variables inside \$MY\_HADOOP\_HOME/bin/setenv.sh are
set correctly. You can set your HADOOP\_HOME, and the locations for your HDFS data
and log directories using this script. You will need to update this script before you can
proceed further.
All the tuning parameters for Hadoop can be found in the \$MY\_HADOOP\_HOME/etc
directory. There is no need to edit any of the parameters, especially if you are not an
expert Hadoop user. If you are familiar with the various Hadoop parameters, you may
edit the parameters that fall outside the “DO NOT EDIT” sections.

\subsection{Request N nodes from the Scheduler}

Once the environment variables have been set correctly, we are ready to use myHadoop
using a regular PBS or SGE submission script. Your PBS script should contain the
following lines to initialize PBS as follows:

\begin{verbatim}
#!/bin/bash
#PBS -q <queue_name>
#PBS -N <job_name>
#PBS -l nodes=4:ppn=1
#PBS -o <output file>
#PBS -e <error_file>
#PBS -A <allocation>
#PBS -V
#PBS -M <user email>
#PBS -m abe
\end{verbatim}

In the above case, we are requesting 4 nodes. Note that you must set the processors per
node (ppn) to 1.
Your SGE script should contain the following lines to initialize SGE:

\begin{verbatim}
#!/bin/bash
#$ -V -cwd
#$ -N <job_name>
#$ -pe <queue_name> 4
#$ -o <output file>
#$ -e <error file>
#$ -S /bin/bash
\end{verbatim}

For SGE, there is one important rule to remember. The queue name specified above
should be preconfigured with an allocation\_rule set to 1 (one). This ensures that the
Hadoop cluster is set up such that multiple instances of the Hadoop daemons are not
scheduled on the same node.

\subsection{Set the myHadoop Environment}

Run the \$MY\_HADOOP\_HOME/bin/setenv.sh script (that you modified in Section 2.1)
to set all the environment variables required by myHadoop.
. \$MY\_HADOOP\_HOME/bin/setenv.sh
Set the HADOOP\_CONF\_DIR to the directory where Hadoop configs should be
generated – all configuration files for the Hadoop run will be picked up from here. Ensure
that this directory is accessible to all nodes – and a way to do this is to make sure that this
directory is on a shared file system such as NFS or Lustre.
export HADOOP\_CONF\_DIR=<configuration directory>

\subsection{Configure the myHadoop Cluster}

You can initialize and configure the Hadoop cluster by using the
\$MY\_HADOOP\_HOME/bin/pbs-configure.sh (or sge-configure.sh) script. You may
create a transient or persistent myHadoop cluster by changing the commandline
arguments as follows.
For a transient myHadoop cluster, configure it as follows (replace 4 with the total number
of nodes requested):
\$MY\_HADOOP\_HOME/bin/pbs-configure.sh -n 4 -c \$HADOOP\_CONF\_DIR
In this mode, you will have to copy all of your data into the myHadoop cluster after it is
configured, and copy out the results after the job is complete. All data will be
inaccessible from HDFS once the PBS job is complete.
Alternatively, you may set up a persistent myHadoop cluster by using the –p option, and
setting the BASE\_DIR for HDFS as follows:
\$MY\_HADOOP\_HOME/bin/pbs-configure.sh -n 4 -c \$HADOOP\_CONF\_DIR -p -d
<HDFS BASE\_DIR>
The BASE\_DIR should be on a directory accessible to all nodes, to ensure that the data
will not be cleaned up after job completion. For instance, the BASE\_DIR could be on a
Lustre file system. Note that, if N-node cluster is being created, then the BASE\_DIR
should have directories named 1, 2, … , N. The configuration script sets up symbolic
links from node I to the BASE\_DIR/I directory. When this mode is used, there is no need
to copy data back and forth from HDFS to another file system between runs.

\subsection{Format HDFS (if need be)}

If myHadoop is being used in transient mode, or if it is being used for the first time in
persistent mode, then you will have to format the HDFS as follows:
\$HADOOP\_HOME/bin/hadoop --config \$HADOOP\_CONF\_DIR namenode –format

\subsection{Run Hadoop Jobs}

You are now all set to start all the Hadoop daemons as follows:
\$HADOOP\_HOME/bin/start-all.sh
Once the daemons are all started up, you can start using Hadoop as usual. You may also
stage data in and out from HDFS, as required.

\subsection{Clean up}

Although, PBS or SGE may be set up to automatically clean up after your Hadoop job is
complete, it is always a good idea to stop all the Hadoop daemons, and use the cleanup
script to clean up after yourself.
\$HADOOP\_HOME/bin/stop-all.sh
\$MY\_HADOOP\_HOME/bin/pbs-cleanup.sh -n 4 OR
\$MY\_HADOOP\_HOME/bin/sge-cleanup.sh -n 4


\subsection{Hadoop}

My Hadoop

We have various platforms that support Hadoop on FutureGrid. MyHadoop is probably the easiest solution offered for you. It provides the advantage that it is integrated into the queuing system and allows hadoop jobs to be run as batch job. This is of especial interest for classes that may run quickly out of resources if every students wants to run their hadoop application at the same time.



MapReduce is a programming model developed by Google. Their definition of MapReduce is as follows: “MapReduce is a programming model and an associated implementation for processing and generating large data sets. Users specify a map function that processes a key/value pair to generate a set of intermediate key/value pairs, and a reduce function that merges all intermediate values associated with the same intermediate key.” For more information about MapReduce, please see the Google paper here.

The Apache Hadoop Project provides an open source implementation of MapReduce and HDFS (Hadoop Distributed File System).

This tutorial illustrates how to run Apache Hadoop thru the batch systems on FutureGrid using the MyHadoop tool.

\subsubsection{myHadoop on FutureGrid}

MyHadoop is a set of scripts that configure and instantiate Hadoop as a batch job.

myHadoop 0.20.2 is currently installed on Alamo, Hotel, India, and Sierra FutureGrid systems.



\FILE{cloudmesh.tex}

\section{Cloudmesh}\label{S:cloudmesh}

\subsection{Functionality}

\begin{figure}[h!]
  \centering
    \includegraphics[width=1.0\textwidth]{images/cm-functionality.pdf}
  \caption{CM Functionality.}
\end{figure}

\subsection{Architecture}

\begin{figure}[h!]
  \centering
    \includegraphics[width=1.0\textwidth]{images/cm-arch.pdf}
  \caption{CM Architecture.}
\end{figure}

\subsection{Provisioning}

\subsection{Cloud Shifting}

\begin{figure}[h!]
  \centering
    \includegraphics[width=1.0\textwidth]{images/shift2.pdf}
  \caption{Shift.}
\end{figure}

\subsection{Instances}

\begin{figure}[h!]
  \centering
    \includegraphics[width=1.0\textwidth]{images/instances.pdf}
  \caption{Rainbow.}
\end{figure}

\subsection{One Click Deployment}
\begin{figure}[h!]
  \centering
    \includegraphics[width=1.0\textwidth]{images/oneclick.pdf}
  \caption{Rainbow.}
\end{figure}

\subsection{Information Services}

\begin{figure}[h!]
  \centering
    \includegraphics[width=1.0\textwidth]{images/rainbow.pdf}
  \caption{Rainbow.}
\end{figure}




\section*{Acknowledgement}

Some of the text published in this chapter is available form the
FutureGrid portal. The FutureGrid project is funded by the National
Science Foundation (NSF) and is led by Indiana University with
University of Chicago, University of Florida, San Diego Supercomputing
Center, Texas Advanced Computing Center, University of Virginia,
University of Tennessee, University of Southern California, Dresden,
Purdue University, and Grid 5000 as partner sites. This material is
based upon work supported in part by the National Science Foundation
under Grant No. 0910812. If you use FutureGrid and produce a paper or
presentation, we ask you to include the reference
\cite{las2010gce,las12fg-bookchapter}.

\bibliographystyle{IEEEtranS}
\bibliography{%
bib/references,%
bib/vonLaszewski-jabref,%
bib/image-refs,%
bib/cyberaide-cloud,%
bib/python}

\appendix

\FILE{overview.tex}

\section{Overview}

FutureGrid is a national-scale Grid, Cloud and HPC computing test-bed service of modest size that includes a number of computational resources at five distributed locations. FutureGrid experience and architecture is built around software defined systems at all levels of the stack shown in figure – encompassing VM and bare-metal infrastructure, networks and application, systems and platform software – with a unifying goal of providing Computing Testbeds as a Service. FutureGrid systems total 4704 cores divided into distributed general purpose clusters at Chicago, Florida, IU and TACC; a Cray XT5m at IU and four small specialized clusters supporting SSD (at SDSC), Large Disk Large memory (at IU) and NVIDIA GPU’s (IU). FutureGrid’s system model has grown in sophistication and now supports software-defined systems – encompassing virtualized and bare-metal infrastructure, networks, application, systems and platform software – with a unifying goal of providing Cloud Testbeds as a Service (CTaaS). Cloudmesh aggregates resources not only from FutureGrid, but also from OpenCirrus, Amazon, Microsoft Azure, and HP Cloud and GENI resources. Cloudmesh was originally developed in order to simplify the execution of multiple concurrent experiments on a federated cloud infrastructure and in addition to virtual resources, FutureGrid exposes bare-metal provisioning to users.

Users can apply for projects either at FutureGrid or XSEDE portals. A single account provides a user access to all FutureGrid machines and we can describe the type of work that can be done by looking at FutureGrid’s experience based on 3 and a half years of operation with 355 projects and 2178 users from 55 countries with 76\% from the USA. 51.8\% of these projects were mainly computer science (including middleware and cyberinfrastructures) research, 9.6\% technology evaluation, 13.2\% education, 9.6\% in life sciences and 11.5\% in other application domains. We looked in detail at the last 200 projects starting October 25 2011 to understand in more detail usage paradigms. 98 of these projects needed only virtual machine (VM) access and 54 requested both virtualized and non-virtualized nodes. Of the 48 projects not requesting VM’s, 8 were studying cloud technology like Hadoop and so 160 projects (80\%) were cloud related. 16 projects involved GPU access and 30\% of all projects used MapReduce in some way. The use of FutureGrid for education has been increasing and 21\% of projects in last 2 years have been for education dividing into 29 semester length classes and the 13 remaining split between REU training, Summer Schools, Tutorials, and workshops. Of 42 education requests, 36 were computer science, 3 application oriented and 3 mixed. These education classes covered areas like cloud computing, distributed systems, parallel computing, big data, data-intensive computing and datamining, business analytics, autonomic computing, cyberinfrastructure, storage, software carpentry, data centers and large scale infrastructure, MapReduce, high performance computing, networking, science clouds, and particular tools supported on FutureGrid.
Coming to the 136 research projects, 109 had a major CS component and 44 an application component with 17 of these jointly classified. Application projects included 18 from bioinformatics including genomics, radiology, cardiovascular simulation, surgery control, health sensors, iPlant cyberinfrastructure and text mining. Only 10 application projects had a simulation (major focus of most HPC systems) focus including combustion, CFD, subsurface modeling, climate, weather, ocean, environment, earthquakes and supply chains. Physical science and engineering data intensive applications (8 projects) include astronomy, particle physics, aerospace reliability, ocean observation, hydroinformatics, GIS and accelerator control. 6 social science projects include conflict resolution, disaster management using Twitter, optimization, political science and economics.

Turning to CS related projects, 23 were in basic virtualization (IaaS) areas. They included provisioning, deployment, new hypervisors including increased performance, elasticity and scheduling, resource management, benchmarking, emulation, and IaaS scaling to support MOOC’s. 4 projects studied distributed clouds and storage with federation. 3 projects centered on networking (routing, optimized devices and emulation) and 2 studied software engineering for clouds. 5 projects were aimed at cyberphysical systems with health, power and mobile applications and FutureGrid used for control and support analytics. 2 projects had a P2P focus covering security and fault tolerance. Security was popular with 10 projects covering trusted P2P storage, mobile and other clients, intrusion detection, file sharing, confidentiality and integrity of data, vulnerability, watermarks and hybrid clouds with different security models. There were 4 HPC programming language projects and 8 covering cloud programming with scheduling, fault tolerance and runtime. 5 fault tolerance projects covered P2P, MapReduce, workflow and HPC. 4 projects covered distributed software transactional memory and concurrency control. Data systems were very popular with 13 projects covering cloud storage, data transfer, NoSQL, streaming big data, Apache software stack, analytics, image retrieval, testing, provenance and the semantic web. 2 projects studied enterprise software issues. 9 artificial intelligence projects covered learning networks for images and social media, large scale image classification by clustering, machine learning, text mining, and agents. Network science with 3 projects saw use of NoSQL datastores to study Twitter, a major infrastructure to support graph and other tools and community detection. Finally 13 projects focused on cyberinfrastructure middleware with XSEDE and EMI (European Middleware initiative) systems, software as a service for HPC simulations, HPC clouds, MPI porting, fault tolerance, workflow, scheduling and resource management, tools, logging and performance.
There were 19 Evaluation projects which covered XSEDE testing (a major actual and intended use of FutureGrid), Open Science Grid testing, particle physics, Apache Big Data stack, Solid State disks, comparison of different VM frameworks, familiarization with cloud technology, GPU’s and studies aimed at planning institutional initiatives in cloud computing. There were 3 Interoperability projects involving long term support of standard end-points. 

FutureGrid interacts with XSEDE on integrating accounting approaches, EOT and software testing.


\end{document}
